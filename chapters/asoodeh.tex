\chapter{چارچوب انقباض $E_\gamma$-واگرایی}
\label{ch:egamma}

\section{مقدمه}
\label{sec:egamma:intro}

در فصل پیشین (\ref{ch:minimax-theory})، ما از واگرایی‌های کلاسیک مانند کولبک-لایبلر (\lr{KL}) و کای-دو ($\chi^2$) برای تحلیل مکانیزم‌های محرمانگی تفاضلی موضعی (\LDP) استفاده کردیم. دیدیم که شرط $\alpha$-\LDP\ باعث می‌شود که واگرایی بین توزیع‌های خروجی با ضریبی از $(e^\alpha - 1)^2$ محدود شود (بخش \ref{sec:privacy-contraction}). این کران‌ها، اگرچه برای اثبات نرخ‌های همگرایی مجانبی (وقتی $n \to \infty$ و $\alpha \to 0$) بسیار مفید هستند، اما دو محدودیت اساسی دارند:

\begin{enumerate}
    \item \textbf{عدم دقت در رژیم‌های میانی:} کران‌های مبتنی بر \lr{KL} معمولاً برای مقادیر بزرگ $\alpha$ (رژیم محرمانگی پایین) بسیار شل\LTRfootnote{Loose} هستند و رفتار دقیق مکانیزم را بازتاب نمی‌دهند.
    \item \textbf{عدم هم‌ارزی:} رابطه بین \LDP\ و واگرایی \lr{KL} یک‌طرفه است؛ یعنی \LDP\ انقباض \lr{KL} را نتیجه می‌دهد، اما انقباض \lr{KL} لزوماً تضمین‌کننده \LDP\ نیست.
\end{enumerate}

در این فصل، ما به سراغ ابزاری می‌رویم که از نظر هندسی و جبری، «هم‌زبان» با تعریف \LDP\ است. بر اساس پژوهش‌های آسوده و همکاران \cite{asoodeh2021local}، نشان می‌دهیم که واگرایی خاصی به نام $E_\gamma$-واگرایی\LTRfootnote{$E_\gamma$-Divergence} (که گاهی واگرایی چوب هاکی\LTRfootnote{Hockey-stick Divergence} نیز نامیده می‌شود)، ابزار طبیعی برای تحلیل \LDP\ است. هدف اصلی این فصل اثبات این حقیقت است که \LDP\ معادل با انقباض کامل (صفر شدن) این واگرایی برای یک مقدار مشخص $\gamma$ است.

\section{تعریف و ویژگی‌های $E_\gamma$-واگرایی}
\label{sec:egamma:def}

واگرایی $E_\gamma$ خانواده‌ای از $f$-واگرایی‌ها است که معیاری برای سنجش «خروج جرم احتمال» یک توزیع نسبت به مضربی از توزیع دیگر ارائه می‌دهد. این معیار در نظریه اطلاعات و آمار بیزی کاربردهای فراوانی دارد.

\begin{تعریف}[$E_\gamma$-واگرایی]
\label{def:egamma}
فرض کنید $P$ و $Q$ دو اندازه احتمال روی فضای اندازه‌پذیر $(\Zset, \mathcal{F})$ باشند و $\gamma \ge 1$ یک اسکالر حقیقی باشد. $E_\gamma$-واگرایی بین $P$ و $Q$ به صورت زیر تعریف می‌شود:
\begin{equation}
\label{eq:egamma-def}
E_\gamma(P \| Q) \triangleq \sup_{S \in \mathcal{F}} \left( P(S) - \gamma Q(S) \right)
\end{equation}
اگر چگالی‌های $p$ و $q$ نسبت به یک اندازه پایه $\nu$ موجود باشند، این تعریف معادل فرم انتگرالی زیر است:
\begin{equation}
\label{eq:egamma-integral}
E_\gamma(P \| Q) = \int_{\Zset} \left( p(z) - \gamma q(z) \right)^+ \nu(dz)
\end{equation}
که در آن $(x)^+ = \max\{0, x\}$ بخش مثبت عدد $x$ است.
\end{تعریف}

\subsection{ویژگی‌های اساسی}
این واگرایی دارای خواص هندسی مهمی است که آن را برای تحلیل محرمانگی ایده‌آل می‌سازد:

\begin{itemize}
    \item \textbf{ارتباط با فاصله تغییرات کل (\lr{TV}):}
    در حالت خاص $\gamma=1$، این واگرایی دقیقاً برابر با فاصله تغییرات کل است:
    \begin{equation}
    E_1(P \| Q) = \sup_{S} (P(S) - Q(S)) = \|P - Q\|_{TV}
    \end{equation}
    
    \item \textbf{رفتار نزولی نسبت به $\gamma$:}
    تابع $\gamma \mapsto E_\gamma(P \| Q)$ یک تابع نزولی و محدب است. هر چه $\gamma$ بزرگتر شود، "جریمه" اختصاص داده شده به $Q$ بیشتر شده و واگرایی کمتر می‌شود.
    
    \item \textbf{نمایش هندسی:}
    اگر ناحیه پذیرش آزمون فرضیه بین $P$ و $Q$ را در نظر بگیریم، $E_\gamma$ متناظر با خطای نوع دوم در آزمون نسبت درست‌نمایی است.
\end{itemize}

\section{هم‌ارزی \lr{LDP} و $E_\gamma$-واگرایی}
\label{sec:egamma:equivalence}

در این بخش، قضیه بنیادی هم‌ارزی را بیان می‌کنیم. این قضیه نشان می‌دهد که تعریف \LDP\ که معمولاً به صورت "کران‌دار بودن نسبت احتمالات" بیان می‌شود، دقیقاً معادل با "صفر شدن $E_\gamma$-واگرایی" برای $\gamma = e^\alpha$ است.

\begin{قضیه}[هم‌ارزی \lr{LDP} و انقباض $E_\gamma$]
\label{thm:ldp-egamma-equiv}
یک مکانیزم تصادفی $\mech: \Xset \to \Zset$ در شرط $\alpha$-\LDP\ صدق می‌کند اگر و تنها اگر برای هر جفت ورودی $x, x' \in \Xset$، مقدار $E_\gamma$-واگرایی بین توزیع‌های خروجی با پارامتر $\gamma = e^\alpha$ برابر صفر باشد. به زبان ریاضی:
\begin{equation}
\mech \in \LDP[\alpha] \iff \sup_{x, x' \in \Xset} E_{e^\alpha}(\mech(\cdot|x) \| \mech(\cdot|x')) = 0
\end{equation}
\end{قضیه}

\begin{اثبات}
اثبات این هم‌ارزی مستقیماً از تعاریف و ویژگی‌های انتگرال حاصل می‌شود.

\textbf{جهت اول ($\Rightarrow$):}
فرض کنید $\mech$ یک مکانیزم $\alpha$-\LDP\ باشد. طبق تعریف \LDP\ (رابطه \ref{eq:ldp-def})، برای هر $x, x'$ و هر مجموعه خروجی $S \subseteq \Zset$ داریم:
\begin{equation}
\Pr[\mech(x) \in S] \le e^\alpha \Pr[\mech(x') \in S]
\end{equation}
اگر $P = \mech(\cdot|x)$ و $Q = \mech(\cdot|x')$ باشند، این نامساوی به صورت $P(S) \le e^\alpha Q(S)$ بازنویسی می‌شود. بنابراین:
\begin{equation}
P(S) - e^\alpha Q(S) \le 0
\end{equation}
با گرفتن سوپریمم روی تمام مجموعه‌های $S$ (طبق تعریف $E_\gamma$ در معادله \ref{eq:egamma-def}):
\begin{equation}
E_{e^\alpha}(P \| Q) = \sup_S (P(S) - e^\alpha Q(S)) \le 0
\end{equation}
از آنجا که واگرایی همیشه نامنفی است، نتیجه می‌گیریم $E_{e^\alpha}(P \| Q) = 0$.

\textbf{جهت دوم ($\Leftarrow$):}
فرض کنید برای تمام $x, x'$ داشته باشیم $E_{e^\alpha}(\mech(\cdot|x) \| \mech(\cdot|x')) = 0$.
از تعریف انتگرالی (\ref{eq:egamma-integral}) استفاده می‌کنیم (فرض می‌کنیم چگالی‌های $p$ و $q$ وجود دارند):
\begin{equation}
\int_{\Zset} (p(z|x) - e^\alpha p(z|x'))^+ \nu(dz) = 0
\end{equation}
از آنجا که عبارت داخل پرانتز $( \cdot )^+$ همواره نامنفی است، انتگرال آن تنها زمانی صفر می‌شود که خود تابع تقریباً همه‌جا (a.e.) صفر باشد. یعنی برای تمام $z \in \Zset$:
\begin{equation}
(p(z|x) - e^\alpha p(z|x'))^+ = 0 \implies p(z|x) - e^\alpha p(z|x') \le 0
\end{equation}
که بازآرایی آن منجر به رابطه زیر می‌شود:
\begin{equation}
\frac{p(z|x)}{p(z|x')} \le e^\alpha
\end{equation}
این دقیقاً همان تعریف شرط $\alpha$-\LDP\ است.
\end{اثبات}

\subsection{تفسیر بر اساس ضریب انقباض}
در ادبیات موضوع \cite{asoodeh2021local}، این نتیجه گاهی با استفاده از مفهوم «ضریب انقباض» بیان می‌شود. ضریب انقباض واگرایی $E_{e^\alpha}$ برای مکانیزم $\mech$ به صورت زیر است:
\begin{equation}
\eta_{E_{e^\alpha}}(\mech) = \sup_{P \neq Q} \frac{E_{e^\alpha}(\mech P \| \mech Q)}{E_{e^\alpha}(P \| Q)}
\end{equation}
قضیه \ref{thm:ldp-egamma-equiv} بیان می‌کند که مکانیزم $\mech$ خصوصی است اگر صورت این کسر برای هر ورودی دیراک (نقطه‌ای) صفر شود. به بیان دیگر، \LDP\ معادل است با وضعیتی که مکانیزم تمامی $E_{e^\alpha}$-واگرایی موجود در ورودی را «از بین ببرد» و به صفر برساند. این نگاه، $E_\gamma$ را از سایر واگرایی‌ها (مانند KL که تنها کاهش می‌یابند اما صفر نمی‌شوند) متمایز می‌کند.


\section{کران‌های انقباض برای $f$-واگرایی‌های عمومی}
\label{sec:contraction-general}

با اثبات هم‌ارزی میان شرط \LDP\ و صفر شدن $E_\gamma$-واگرایی (قضیه \ref{thm:ldp-egamma-equiv})، اکنون ابزاری قدرتمند برای بازنویسی کران‌های انقباض در اختیار داریم. استراتژی اصلی در این بخش مبتنی بر این واقعیت است که هر $f$-واگرایی محدب را می‌توان به صورت انتگرالی از $E_\gamma$-واگرایی‌ها نوشت. بنابراین، اگر ما کنترلی روی $E_\gamma$ داشته باشیم (که به واسطه \LDP\ داریم)، می‌توانیم آن را به سایر واگرایی‌ها تعمیم دهیم.

\subsection{تجزیه طیفی $f$-واگرایی‌ها}
طبق نتایج موجود در نظریه اطلاعات (رجوع کنید به \cite{asoodeh2021local})، هر $f$-واگرایی برای توزیع‌های $P$ و $Q$ می‌تواند به فرم زیر نمایش داده شود:
\begin{equation}
D_f(P \| Q) = \int_1^\infty E_\gamma(P \| Q) d\mu_f(\gamma)
\end{equation}
که در آن $\mu_f$ یک اندازه (Measure) است که بستگی به تابع مولد $f$ دارد. این نمایش به ما اجازه می‌دهد تا ویژگی‌های انقباضی $E_\gamma$ را به طور مستقیم به $D_f$ منتقل کنیم.

\subsection{قضیه انقباض بهبودیافته برای $\chi^2$}
یکی از مهم‌ترین نتایج این چارچوب جدید، بهبود کران‌های انقباض برای واگرایی کای-دو ($\chi^2$) است. همان‌طور که در فصل ۴ (قضیه \ref{thm:chi-squared-contraction}) دیدیم، تحلیل‌های کلاسیک دوچی و همکاران کرانی با ضریب $(e^\alpha + 1)(e^\alpha - 1)^2$ ارائه می‌دادند. در اینجا، با استفاده از تکنیک‌های مقاله ۲۰۲۴ آسوده و ژانگ \cite{asoodeh2024contraction}، کران بسیار دقیق‌تری (تنگ‌تر) را بیان می‌کنیم.

\begin{قضیه}[کران دقیق انقباض $\chi^2$]
\label{thm:sharp-chi-square}
فرض کنید $\mech$ یک مکانیزم $\alpha$-\LDP\ باشد. برای هر جفت توزیع ورودی $P_1$ و $P_2$، واگرایی $\chi^2$ بین توزیع‌های خروجی $M_1 = \mech(P_1)$ و $M_2 = \mech(P_2)$ به صورت زیر کران‌دار می‌شود:
\begin{equation}
\label{eq:sharp-chi-bound}
D_{\chi^2}(M_1 \| M_2) \le (e^\alpha - 1)^2 \|P_1 - P_2\|_{TV}^2
\end{equation}
همچنین، یک کران کلی‌تر برای هر $f$-واگرایی به صورت زیر برقرار است (قضیه ۳ در \cite{asoodeh2024contraction}):
\begin{equation}
D_f(M_1 \| M_2) \le \eta_f(\mech) \cdot D_f(P_1 \| P_2)
\end{equation}
که در آن ضریب انقباض برای مکانیزم‌های باینری بهینه برابر است با $\eta_{\chi^2} = \tanh^2(\frac{\alpha}{2})$.
\end{قضیه}

\begin{اثبات}
اثبات دقیق این قضیه نیازمند تحلیل مقادیر ویژه عملگرهای مارکوف است که در \cite{asoodeh2024contraction} آمده است. ایده کلی این است که چون $E_{e^\alpha}(M_1 \| M_2) = 0$ است، جرم احتمال در «دنباله» توزیع‌ها (جایی که نسبت $p/q$ بزرگ است) کاملاً حذف می‌شود. با استفاده از نامساوی‌های پردازش داده قوی (\lr{SDPI}) که مستقیماً از هم‌ارزی $E_\gamma$ استخراج شده‌اند، می‌توان نشان داد که ضریب $(e^\alpha+1)$ که در تحلیل‌های قبلی ظاهر می‌شد، ناشی از تقریب‌های نادقیق بوده و قابل حذف است.
\end{اثبات}

\subsection{مقایسه با کران‌های دوچی (فصل ۴)}
برای درک اهمیت قضیه \ref{thm:sharp-chi-square}، بیایید ضریب به دست آمده را با نتیجه دوچی (رابطه \ref{eq:chi-sq-bound} در فصل ۴) مقایسه کنیم.

\begin{itemize}
    \item \textbf{کران دوچی (Duchi et al. 2013):}
    $$ D_{\chi^2}(M_1 \| M_2) \le \underbrace{(e^\alpha + 1)(e^\alpha - 1)^2}_{\approx e^{3\alpha}} \|P_1 - P_2\|_{TV}^2 $$
    این کران در رژیم محرمانگی پایین ($\alpha$ بزرگ) با سرعت $e^{3\alpha}$ رشد می‌کند.
    
    \item \textbf{کران بهبودیافته (Asoodeh et al. 2024):}
    $$ D_{\chi^2}(M_1 \| M_2) \le \underbrace{(e^\alpha - 1)^2}_{\approx e^{2\alpha}} \|P_1 - P_2\|_{TV}^2 $$
    این کران با سرعت $e^{2\alpha}$ رشد می‌کند.
\end{itemize}

\textbf{تحلیل بهبود:}
تفاوت این دو کران در ضریب $(e^\alpha + 1)$ است.
\begin{enumerate}
    \item در رژیم $\alpha \to 0$ (محرمانگی بالا)، $e^\alpha + 1 \to 2$، بنابراین کران جدید تا ضریب ۲ بهتر است (که در تحلیل‌های مجانبی تأثیر کمی دارد).
    \item اما در رژیم $\alpha > 1$ (که در کاربردهای عملی بسیار رایج است)، حذف عامل نمایی $e^\alpha$ از ضریب، بهبودی چشمگیر است. این نشان می‌دهد که تحلیل‌های مبتنی بر $E_\gamma$، رفتار مکانیزم را در تمام بازه‌های $\alpha$ بسیار دقیق‌تر توصیف می‌کنند و از برآوردهای بیش‌از‌حد بدبینانه (Over-pessimistic) جلوگیری می‌کنند.
\end{enumerate}

این کران‌های دقیق‌تر، مستقیماً منجر به استخراج نرخ‌های مینی‌مکس دقیق‌تر در مسائل تخمین توزیع و آزمون فرض می‌شوند که در ادامه فصل به آن‌ها خواهیم پرداخت.


\section{کاربرد در نرخ‌های مینی‌مکس و مقایسه نهایی}
\label{sec:egamma:applications}

در بخش‌های پیشین، ما ابزارهای قدرتمندی را برای تحلیل مکانیزم‌های \LDP\ توسعه دادیم. اکنون زمان آن رسیده است که این ابزارها را در عمل به کار بگیریم. در این بخش، نشان می‌دهیم که چگونه استفاده از چارچوب انقباض $E_\gamma$ و کران‌های دقیق $\chi^2$، منجر به اثبات نرخ‌های مینی‌مکس دقیق‌تر (تنگ‌تر) نسبت به روش‌های کلاسیک فصل ۴ می‌شود.

\subsection{بازبینی مسئله تخمین میانگین}
مسئله تخمین میانگین $\theta \in [-1, 1]^d$ را که در بخش \ref{sec:minimax:mean-estimation} معرفی شد، در نظر بگیرید. در آنجا با استفاده از لم آسوآد و کران انقباض کلاسیک دوچی، به نرخ زیر رسیدیم:
\begin{equation}
\mathfrak{M}_n(\LDP) \ge c \cdot \frac{d}{n \alpha^2}
\end{equation}
نکته کلیدی در اثبات آن بخش، استفاده از کران انقباضی بود که شامل ضریب $(e^\alpha + 1)$ می‌شد. حال بیایید ببینیم با ابزارهای جدید چه تغییری در این تحلیل ایجاد می‌شود.

طبق روش لم آسوآد (رابطه \ref{eq:assouad-step1})، بخش اصلی اثبات، کران‌دار کردن مجموع احتمال‌های خطا در تشخیص بین دو توزیع همسایه $P_v$ و $P_{v'}$ است. دیدیم که این خطا با فاصله تغییرات کل توزیع‌های خروجی $\|M_v^n - M_{v'}^n\|_{TV}$ مرتبط است.
با استفاده از نامساوی‌های پردازش داده قوی (\lr{SDPI}) و کران دقیق $\chi^2$ که در قضیه \ref{thm:sharp-chi-square} ثابت کردیم، داریم:
\begin{align}
\|M_v - M_{v'}\|_{TV}^2 &\le \frac{1}{4} D_{\chi^2}(M_v \| M_{v'}) \nonumber \\
&\le \frac{1}{4} \eta_{\chi^2}(\mech) \cdot D_{\chi^2}(P_v \| P_{v'})
\end{align}
که در آن $\eta_{\chi^2}(\mech)$ ضریب انقباض دقیق مکانیزم است. بر اساس نتایج آسوده و همکاران \cite{asoodeh2024contraction}، برای مکانیزم‌های بهینه باینری، این ضریب دقیقاً برابر است با:
\begin{equation}
\eta_{\chi^2}^*(\alpha) = \tanh^2\left(\frac{\alpha}{2}\right) = \left( \frac{e^{\alpha/2} - e^{-\alpha/2}}{e^{\alpha/2} + e^{-\alpha/2}} \right)^2 = \left( \frac{e^\alpha - 1}{e^\alpha + 1} \right)^2
\end{equation}
این نتیجه بسیار دقیق‌تر از کران‌های پیشین است. بیایید رفتار این ضریب را در دو رژیم حدی بررسی کنیم:

\begin{enumerate}
    \item \textbf{رژیم محرمانگی بالا ($\alpha \to 0$):}
    $$ \eta_{\chi^2}^*(\alpha) \approx \left(\frac{\alpha}{2}\right)^2 = \frac{\alpha^2}{4} $$
    این نتیجه با تحلیل‌های فصل ۴ (که $\alpha^2$ بود) سازگار است. این نشان می‌دهد که تحلیل‌های دوچی در رژیم $\alpha$ کوچک، مجانباً بهینه بوده‌اند.
    
    \item \textbf{رژیم محرمانگی پایین ($\alpha \to \infty$):}
    در این حالت، $\eta_{\chi^2}^* \to 1$. این رفتار بسیار منطقی است؛ وقتی محرمانگی وجود ندارد، انقباضی رخ نمی‌دهد. اما کران‌های کلاسیک فصل ۴ در این حالت به بی‌نهایت میل می‌کردند ($\approx e^{3\alpha}$) که از نظر فیزیکی بی‌معنی بود. چارچوب جدید به درستی پیش‌بینی می‌کند که با افزایش بودجه محرمانگی، اطلاعات از دست رفته به صفر میل می‌کند.
\end{enumerate}

با جایگذاری این ضریب دقیق در لم آسوآد، نرخ مینی‌مکس به صورت دقیق‌تری بیان می‌شود:
\begin{equation}
\mathfrak{M}_n \ge \frac{d}{n \cdot \tanh^2(\alpha/2)}
\end{equation}
این فرمول نه تنها برای $\alpha$های کوچک، بلکه برای تمام طیف پارامترهای محرمانگی معتبر است.

\section{نتیجه‌گیری فصل}
\label{sec:egamma:conclusion}

در این فصل، ما یک گام بنیادین فراتر از تحلیل‌های استاندارد \lr{LDP} برداشتیم. هدف ما گذار از «تقریب‌های جبری» به «تعاریف هندسی دقیق» بود. دستاوردهای اصلی این فصل را می‌توان در سه محور خلاصه کرد:

\begin{itemize}
    \item \textbf{هم‌ارزی ذاتی:} ما نشان دادیم که \lr{LDP} صرفاً یک محدودیت روی نسبت‌های احتمال نیست، بلکه دقیقاً معادل با صفر شدن $E_\gamma$-واگرایی (برای $\gamma=e^\alpha$) است. این یعنی $E_\gamma$، متریک طبیعی و ذاتی فضای \LDP\ است، همان‌طور که فاصله اقلیدسی متریک طبیعی فضای برداری $\mathbb{R}^n$ است.
    
    \item \textbf{انقباض دقیق:} با استفاده از تجزیه طیفی $f$-واگرایی‌ها بر حسب $E_\gamma$، توانستیم نشان دهیم که مکانیزم‌های خصوصی چقدر اطلاعات را «منقبض» می‌کنند. ما ثابت کردیم که بسیاری از ضرایب نمایی $e^\alpha$ که در مقالات پیشین (مانند \cite{Duchi2013}) وجود داشتند، ناشی از ابزار نامناسب تحلیل (استفاده از \lr{KL} به جای $E_\gamma$) بوده‌اند و قابل حذف هستند.
    
    \item \textbf{وحدت رویه:} این چارچوب جدید، یک روش واحد برای تحلیل انواع مسائل (تخمین، آزمون فرض، و یادگیری) فراهم می‌کند. به جای استفاده از تکنیک‌های مجزا برای هر واگرایی، ما تنها رفتار $E_\gamma$ را بررسی می‌کنیم و نتایج را به سایر معیارها تعمیم می‌دهیم.
\end{itemize}

به این ترتیب، فصل ۵ نه تنها نتایج فصل‌های قبل را تأیید می‌کند، بلکه پایه‌ای مستحکم برای طراحی مکانیزم‌های بهینه در آینده فراهم می‌آورد. این دیدگاه هندسی به ما می‌گوید که یک مکانیزم \LDP\ بهینه، مکانیزمی است که دقیقاً روی مرز انقباض $E_{e^\alpha}$ حرکت کند.