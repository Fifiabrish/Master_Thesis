\chapter{نتیجه‌گیری و پیشنهادها}
\label{ch:conclusion}

\section{جمع‌بندی و دستاوردهای اصلی}
\label{sec:conclusion:summary}

در این پایان‌نامه، ما مسیر پرفراز و نشیبی را برای درک حدود نهایی دقت آماری در حضور محدودیت‌های محرمانگی طی کردیم. هدف اصلی ما، پاسخ به این پرسش بود که «هزینه اطلاعاتی» واقعی که محرمانگی تفاضلی موضعی (\LDP) بر داده‌ها تحمیل می‌کند، چیست و چگونه می‌توان آن را به دقیق‌ترین شکل ممکن مدل‌سازی کرد.

مسیر پژوهش ما به شرح زیر طی شد:
\begin{itemize}
    \item در \textbf{فصول دوم و سوم}، زیربنای ریاضی لازم را بنا نهادیم. ما تعریف \LDP\ را نه صرفاً به عنوان یک الگوریتم، بلکه به عنوان یک محدودیت روی کانال‌های ارتباطی معرفی کردیم و ابزارهای سنجش فاصله ($f$-واگرایی‌ها) را مرور نمودیم.
    
    \item در \textbf{فصل چهارم}، رویکرد کلاسیک و پیشگامانه دوچی و همکاران \cite{Duchi2013} را بررسی کردیم. دیدیم که در این رویکرد، تحلیل‌ها بر مبنای «انقباض واگرایی \lr{KL}» بنا شده‌اند. اگرچه این روش برای رژیم‌های محرمانگی بالا ($\alpha \to 0$) نتایج قابل قبولی ارائه می‌دهد، اما نشان دادیم که استفاده از \lr{KL} برای تحلیل \LDP\ یک انتخاب ناگزیر بوده و نه یک انتخاب طبیعی؛ زیرا \lr{KL} تقارن و کران‌های ذاتی \LDP\ را به طور کامل بازتاب نمی‌دهد و منجر به ظهور ضرایب نادقیق در کران‌های مینی‌مکس می‌شود.
    
    \item در \textbf{فصل پنجم} (که هسته اصلی نوآوری این پژوهش بود)، ما چارچوب نوین «انقباض $E_\gamma$-واگرایی» را معرفی کردیم. ما نشان دادیم که واگرایی $E_\gamma$ (با پارامتر $\gamma = e^\alpha$) زبان مادری و طبیعی \LDP\ است. اثبات کردیم که \LDP\ معادل با صفر شدن کامل این واگرایی است و این دیدگاه هندسی، ما را قادر ساخت تا کران‌های انقباض را برای سایر معیارها (مانند $\chi^2$) اصلاح و دقیق‌سازی کنیم.
\end{itemize}

\subsection{دستاوردهای اصلی پژوهش}
مهم‌ترین دستاورد این پایان‌نامه، «یکپارچه‌سازی تحلیل \LDP» از طریق هندسه $E_\gamma$ است. نتایج کلیدی ما عبارتند از:

\begin{enumerate}
    \item \textbf{گذار از تقریب به دقت کامل:} 
    در روش‌های سنتی (فصل ۴)، برای استخراج کران‌های پایین، ناچار به استفاده از بسط‌های تیلور پیچیده حول صفر برای واگرایی \lr{KL} بودیم. این بسط‌ها تنها برای $\alpha$های کوچک معتبر بودند. در مقابل، رویکرد فصل ۵ نشان داد که با استفاده از $E_\gamma$، می‌توانیم بدون نیاز به بسط تیلور و به صورت مستقیم، کران‌هایی را استخراج کنیم که برای تمام مقادیر $\alpha \in (0, \infty)$ معتبر و دقیق هستند.
    
    \item \textbf{اصلاح ضرایب انقباض:}
    ما نشان دادیم که ضریب انقباض واقعی برای مکانیزم‌های \LDP\ از مرتبه $(e^\alpha - 1)^2$ است، در حالی که تحلیل‌های قبلی دارای ضریب اضافه $(e^\alpha + 1)$ بودند. این اصلاح، به ویژه در کاربردهایی که بودجه محرمانگی $\alpha$ مقداری متوسط یا بزرگ دارد (مانند یادگیری ماشین فدرال)، تفاوتی معنادار در تخمین حجم داده مورد نیاز ایجاد می‌کند.
    
    \item \textbf{دیدگاه هندسی به جای جبری:}
    این پژوهش نشان داد که مسئله طراحی مکانیزم‌های بهینه \LDP، در واقع مسئله یافتن توزیع‌هایی است که $E_{e^\alpha}$-واگرایی آن‌ها صفر باشد. این دیدگاه هندسی (قرار گرفتن در همسایگی مشخصی نسبت به متریک $E_\gamma$)، درک عمیق‌تری نسبت به تعریف جبری «نسبت احتمالات» فراهم می‌کند و راه را برای طراحی مکانیزم‌های جدید هموار می‌سازد.
\end{enumerate}

به طور خلاصه، این پایان‌نامه نشان داد که برای تحلیل دقیق سیستم‌های خصوصی، باید ابزار تحلیل (متریک واگرایی) را با ماهیت محدودیت (شرط محرمانگی) همراستا کرد. انتخاب $E_\gamma$ به عنوان این ابزار، شفافیت و دقت تحلیل‌های آماری در حوزه محرمانگی را به سطح جدیدی ارتقا داده است.


\section{پیشنهادهایی برای تحقیقات آتی}
\label{sec:future-works}

پژوهش حاضر با معرفی چارچوب انقباض $E_\gamma$-واگرایی، گامی بنیادین در جهت یکپارچه‌سازی نظریه محرمانگی تفاضلی موضعی برداشت. با این حال، همچنان چالش‌های نظری و عملی بسیاری در این حوزه باقی مانده است. در ادامه، چهار مسیر پژوهشی اصلی را که می‌توانند به عنوان امتداد طبیعی این پایان‌نامه مورد توجه قرار گیرند، معرفی می‌کنیم.

\subsection{تحلیل انقباض در مدل شافل (\lr{Shuffle Model})}
در سال‌های اخیر، «مدل شافل» به عنوان یک حد واسط میان مدل‌های \LDP\ و \CDP\ ظهور کرده است. در این مدل، داده‌های خصوصی‌سازی شده توسط کاربران، پیش از رسیدن به تحلیل‌گر، توسط یک واسط امن (شافل‌کننده) به صورت تصادفی جایگشت داده می‌شوند.
ادبیات موجود نشان می‌دهد که عمل شافل کردن باعث «تقویت محرمانگی»\LTRfootnote{Privacy Amplification} می‌شود. یک مسئله باز جذاب، بررسی این پدیده از دیدگاه $E_\gamma$-واگرایی است.
\begin{itemize}
    \item \textbf{پرسش:} آیا می‌توان نشان داد که عملگر شافل، ضریب انقباض $E_\gamma$ را با نرخی بهتر از تحلیل‌های موجود بهبود می‌بخشد؟ استفاده از هندسه $E_\gamma$ ممکن است کران‌های دقیق‌تری برای پدیده تقویت محرمانگی نسبت به واگرایی‌های تقریبی \lr{KL} ارائه دهد.
\end{itemize}

\subsection{ارتباط با محرمانگی تفاضلی رنی (\lr{RDP})}
محرمانگی تفاضلی رنی (\lr{RDP}) که بر مبنای واگرایی‌های رنی بنا شده است، امروزه استاندارد طلایی برای تحلیل ترکیب مکانیزم‌ها (به‌ویژه در یادگیری عمیق و الگوریتم \lr{DP-SGD}) محسوب می‌شود.
در فصل ۵ دیدیم که $E_\gamma$ ارتباط مستقیمی با \LDP\ دارد. از سوی دیگر، می‌دانیم که واگرایی‌های رنی نیز خانواده‌ای از $f$-واگرایی‌ها هستند.
\begin{itemize}
    \item \textbf{پرسش:} چه رابطه‌ی صریحی میان انقباض $E_\gamma$ و انقباض واگرایی‌های رنی وجود دارد؟ استخراج فرمول‌های تبدیل بین این دو، می‌تواند به ما اجازه دهد تا از ابزارهای قدرتمند \lr{RDP} (مانند مومنت‌های ترکیبی) در چارچوب دقیق $E_\gamma$ استفاده کنیم.
\end{itemize}

\subsection{تخمین‌گرهای تطبیقی (\lr{Adaptive Estimators})}
در تحلیل‌های مینی‌مکس این پایان‌نامه، فرض بر این بود که بودجه محرمانگی $\alpha$ ثابت است و مکانیزم $\mech$ مستقل از داده عمل می‌کند. با این حال، در بسیاری از کاربردهای عملی، داده‌ها دارای ساختار ناهمگن هستند (مثلاً واریانس داده‌ها در نواحی مختلف دامنه متفاوت است).
\begin{itemize}
    \item \textbf{پرسش:} آیا می‌توان مکانیزم‌هایی طراحی کرد که به صورت «تطبیقی» عمل کنند؟ به این معنا که مکانیزم ابتدا یک تخمین نویزدار از سختی داده (مثلاً واریانس محلی) به دست آورد و سپس $\alpha$ را متناسب با آن تنظیم کند. چالش اصلی در اینجا اثبات این نکته است که فرآیند تنظیم پارامتر، خود باعث نقض شرط \LDP\ نشود. تحلیل انقباض برای چنین مکانیزم‌های دو-مرحله‌ای هنوز به طور کامل توسعه نیافته است.
\end{itemize}

\subsection{تعمیم به داده‌های وابسته (غیر \lr{i.i.d.})}
تمامی کران‌های مینی‌مکس اثبات شده در فصل ۴ و ۵، بر فرض «مستقل و هم‌توزیع بودن» (\lr{i.i.d.}) داده‌ها استوار بودند. این در حالی است که در کاربردهای واقعی \LDP\ (مانند جمع‌آوری داده از حسگرهای اینترنت اشیاء یا رفتارهای کاربر در طول زمان)، داده‌ها دارای همبستگی زمانی یا مکانی هستند.
\begin{itemize}
    \item \textbf{پرسش:} چگونه می‌توان لم آسوآد و نامساوی‌های انقباض $E_\gamma$ را برای فرآیندهای تصادفی وابسته (مانند زنجیره‌های مارکوف) تعمیم داد؟ در حضور وابستگی، خاصیت تانسوری واگرایی‌ها (که $D(P^n \| Q^n) = n D(P \| Q)$) برقرار نیست و نیازمند توسعه ابزارهای جدیدی برای سنجش نرخ انباشت اطلاعات در سیستم‌های دینامیکی خصوصی هستیم.
\end{itemize}