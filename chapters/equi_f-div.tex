\chapter{هم‌ارزی \lr{LDP} و انقباض $E_\gam$-واگرایی}
\label{ch:new-results}

\section{مقدمه و انگیزه}
\label{sec:new:intro}

در فصل پیشین، دیدیم که چگونه دوچی و همکاران \cite{Duchi2013} از واگرایی کولبک-لایبلر (\lr{KL}) برای تحلیل محرمانگی تفاضلی موضعی استفاده کردند. اگرچه کران‌های آن‌ها برای رژیم‌های محرمانگی بالا (\al\ کوچک) بسیار کارآمد هستند، اما در رژیم‌های \al\ متوسط و بزرگ، دقت خود را از دست می‌دهند.

مشکل اصلی در آنجاست که واگرایی \lr{KL} متریک «بومی» برای تعریف \LDP نیست. تعریف \LDP (معادله \ref{eq:ldp-def}) مبتنی بر نسبت احتمالات است، در حالی که \lr{KL} مبتنی بر لگاریتم نسبت‌هاست. این ناهمخوانی باعث می‌شود که در تبدیل شرایط \LDP به کران‌های \lr{KL}، اطلاعاتی از دست برود (\lr{lossy conversion}).

در این فصل، نشان می‌دهیم که یک معیار واگرایی دیگر به نام $E_\gam$-واگرایی وجود دارد که دقیقاً ساختار هندسی \LDP را تسخیر می‌کند. ما ثابت خواهیم کرد که شرط \LDP دقیقاً معادل صفر شدن $E_\gam$-واگرایی (برای $\gam = e^\al$) است. سپس از این هم‌ارزی برای استخراج کران‌های انقباض دقیق\LTRfootnote{Tight} برای سایر واگرایی‌ها استفاده خواهیم کرد که نتایج دوچی را بهبود می‌بخشند \cite{Asoodeh2021}.

\section{معرفی $E_\gam$-واگرایی}
\label{sec:new:egamma-def}

$E_\gam$-واگرایی یکی از اعضای کمتر شناخته‌شده‌ی خانواده \f-واگرایی‌هاست که در نظریه اطلاعات برای مقایسه نسبت درست‌نمایی توزیع‌ها کاربرد دارد.

\begin{تعریف}[$E_\gam$-واگرایی]
فرض کنید $P$ و $Q$ دو توزیع احتمال باشند و $\gam \ge 1$ یک عدد حقیقی باشد. $E_\gam$-واگرایی بین $P$ و $Q$ به صورت زیر تعریف می‌شود:
\begin{equation}
\label{eq:egamma-def}
E_\gam(P || Q) = \sup_{\Sset \in \sigma(\Xset)} \left( P(\Sset) - \gam Q(\Sset) \right)
\end{equation}
\end{تعریف}

این تعریف را می‌توان به صورت بسته‌ی زیر نیز نوشت:
\begin{equation}
E_\gam(P || Q) = \int_{\Xset} \max \{0, p(x) - \gam q(x)\} d\muu(x)
\end{equation}
که در آن $p$ و $q$ توابع چگالی احتمال هستند.

\subsection{خواص هندسی}
این واگرایی خواص جالبی دارد که آن را برای تحلیل محرمانگی ایده‌آل می‌کند:
\begin{itemize}
    \item \textbf{ارتباط با فاصله واریانس کل:} اگر $\gam = 1$ باشد، داریم:
    \begin{equation}
    E_1(P || Q) = \sup_{\Sset} (P(\Sset) - Q(\Sset)) = \norm{P - Q}_{TV}
    \end{equation}
    بنابراین $E_\gam$ تعمیمی از فاصله \lr{TV} است.
    
    \item \textbf{غیرمنفی بودن:} همواره $E_\gam(P || Q) \ge 0$ نیست. در واقع، اگر نسبت $p(x)/q(x)$ همواره کمتر از $\gam$ باشد، این مقدار صفر می‌شود. دقیقاً همین ویژگی است که آن را به \LDP مرتبط می‌کند.
\end{itemize}

\section{قضیه هم‌ارزی اصلی}
\label{sec:new:equivalence}

اکنون به مهم‌ترین نتیجه‌ی نظری این پایان‌نامه می‌رسیم: اثبات اینکه \LDP چیزی جز محدودیت بر روی $E_\gam$-واگرایی نیست.

\begin{قضیه}[هم‌ارزی \LDP و $E_\gam$]
\label{thm:ldp-egamma}
یک مکانیزم \mech\ در شرط \LDP صدق می‌کند اگر و تنها اگر برای تمام جفت ورودی‌های $x, x' \in \Xset$:
\begin{equation}
\label{eq:equivalence}
E_{e^\al}(\mech(\cdot|x) || \mech(\cdot|x')) = 0
\end{equation}
\end{قضیه}

\begin{اثبات}
اثبات را در دو جهت انجام می‌دهیم.

\textbf{جهت اول ($\Rightarrow$):} فرض کنید \mech\ خاصیت \LDP دارد. طبق تعریف \ref{def:alpha-ldp}، برای هر زیرمجموعه خروجی $\Sset \subseteq \Zset$ و هر $x, x'$ داریم:
\begin{equation}
\Pr[\mech(x) \in \Sset] \le e^\al \Pr[\mech(x') \in \Sset]
\end{equation}
این نامساوی را می‌توان به صورت زیر بازنویسی کرد:
\begin{equation}
\Pr[\mech(x) \in \Sset] - e^\al \Pr[\mech(x') \in \Sset] \le 0
\end{equation}
از آن‌جایی که این رابطه برای \textit{تمام} \Sset ها برقرار است، سوپریمم آن نیز باید کوچکتر یا مساوی صفر باشد. اما طبق تعریف $E_\gam$ در معادله \ref{eq:egamma-def}، این سوپریمم دقیقاً همان $E_{e^\al}$ است. چون $E_\gam$ نمی‌تواند منفی باشد (با انتخاب $\Sset=\emptyset$ مقدار حداقل صفر است)، پس حتماً برابر صفر است.

\textbf{جهت دوم ($\Leftarrow$):} فرض کنید $E_{e^\al}(\mech(\cdot|x) || \mech(\cdot|x')) = 0$.
طبق تعریف سوپریمم، برای هر مجموعه دلخواه \Sset:
\begin{equation}
\Pr[\mech(x) \in \Sset] - e^\al \Pr[\mech(x') \in \Sset] \le 0
\end{equation}
که بلافاصله نتیجه می‌دهد:
\begin{equation}
\frac{\Pr[\mech(x) \in \Sset]}{\Pr[\mech(x') \in \Sset]} \le e^\al
\end{equation}
این دقیقاً همان تعریف \LDP است.
\end{اثبات}

این قضیه ساده اما بنیادین، یک تفسیر هندسی دقیق از محرمانگی ارائه می‌دهد: \textbf{\LDP یعنی توزیع‌های خروجی چنان به هم نزدیک باشند که هیچ بخشی از دامنه نتواند نسبت درست‌نمایی بیشتر از $e^\al$ ایجاد کند.}

\section{بهبود کران‌های انقباض}
\label{sec:new:tight-bounds}

در فصل ۳ دیدیم که دوچی \cite{Duchi2013} کران زیر را برای انقباض \lr{KL} ارائه کرد:
\begin{equation}
D_{KL}(\mech(\cdot|x) || \mech(\cdot|x')) \le 4(e^\al - 1)^2
\end{equation}
حال با استفاده از چارچوب $E_\gam$، می‌توانیم کران‌های بسیار دقیق‌تری استخراج کنیم. آسوده و همکاران \cite{Asoodeh2021} نشان داده‌اند که اگر شرط $E_{e^\al} = 0$ برقرار باشد، می‌توان کران‌های انقباض برای سایر \f-واگرایی‌ها را از طریق بهینه‌سازی محدب به دست آورد.

\begin{قضیه}[کران دقیق انقباض \lr{KL}]
اگر \mech\ یک مکانیزم \LDP باشد، آنگاه:
\begin{equation}
D_{KL}(\mech(\cdot|x) || \mech(\cdot|x')) \le \frac{e^\al - 1}{e^\al + 1} \cdot (e^\al - 1)
\end{equation}
برای مقادیر کوچک \al\ (رژیم محرمانگی بالا)، این کران به $\al^2/2$ میل می‌کند که ۴ برابر کوچکتر (بهتر) از کران دوچی است.
\end{قضیه}

\textbf{تحلیل مقایسه‌ای:}
بیایید رفتار دو کران را در $\al \to 0$ بررسی کنیم:
\begin{itemize}
    \item کران دوچی: $4(e^\al - 1)^2 \approx 4\al^2$
    \item کران مبتنی بر $E_\gam$: $\approx \frac{\al}{2} \cdot \al = \frac{\al^2}{2}$ (چون $\frac{e^\al-1}{e^\al+1} \approx \tanh(\al/2) \approx \al/2$)
\end{itemize}
این بهبود ضریب ثابت (از ۴ به $0.5$) در تحلیل‌های مینیماکس بسیار حیاتی است و نشان می‌دهد که «اندازه نمونه موثر» واقعی می‌تواند تا ۸ برابر بهتر از چیزی باشد که آنالیزهای قبلی نشان می‌دادند.

\section{تعمیم به محرمانگی تقریبی (\lr{(\al, \del)-LDP})}
\label{sec:new:approx-ldp}

یکی دیگر از قدرت‌های چارچوب $E_\gam$، توانایی آن در توصیف ساده‌ی محرمانگی تقریبی است.
یادآوری می‌کنیم که \LDP[(\al, \del)] شرط زیر را دارد:
\begin{equation}
\Pr[\mech(x) \in \Sset] \le e^\al \Pr[\mech(x') \in \Sset] + \del
\end{equation}
با بازنویسی این رابطه داریم:
\begin{equation}
\sup_{\Sset} (\Pr[\mech(x) \in \Sset] - e^\al \Pr[\mech(x') \in \Sset]) \le \del
\end{equation}
که دقیقاً معادل است با:
\begin{equation}
E_{e^\al}(\mech(\cdot|x) || \mech(\cdot|x')) \le \del
\end{equation}

\begin{نتیجه}
محرمانگی تقریبی \LDP[(\al, \del)] دقیقاً معادل محدود کردن مقدار $E_{e^\al}$-واگرایی توزیع‌های خروجی به مقدار \del\ است. این نتیجه نشان می‌دهد که $E_\gam$-واگرایی طبیعی‌ترین زبان برای صحبت درباره محرمانگی تفاضلی (چه خالص و چه تقریبی) است.
\end{نتیجه}

\section{کاربرد در تخمین توزیع گسسته}
\label{sec:new:application}

برای نشان دادن کاربرد عملی این نتایج، مسئله تخمین توزیع احتمال روی یک دامنه $k$-تایی را در نظر بگیرید.
با استفاده از تکنیک‌های انقباض $E_\gam$، می‌توان نشان داد که نرخ مینیماکس برای این مسئله تحت شرط \LDP برابر است با:
\begin{equation}
\mathfrak{M}_n \asymp \frac{k}{n (e^\al - 1)^2}
\end{equation}
در حالی که استفاده از تکنیک‌های کلاسیک (دوچی)، جمله‌ای به صورت $\frac{k}{n \al^2}$ را پیشنهاد می‌کرد. تفاوت این دو عبارت در رژیم \al\ بزرگ (محرمانگی کم) آشکار می‌شود؛ جایی که $(e^\al-1)^2$ به صورت نمایی رشد می‌کند و نشان می‌دهد که دقت می‌تواند بسیار سریع‌تر از پیش‌بینی‌های قبلی بهبود یابد.

\section{انقباض قوی برای خانواده‌ی \f-واگرایی‌ها}
\label{sec:new:strong-contraction}

تا اینجا دیدیم که شرط \LDP معادل صفر شدن $E_{e^\al}$-واگرایی است. یک پرسش طبیعی و بسیار مهم این است: آیا این شرط بر روی سایر معیارهای فاصله (مثل $\ch^2$ یا هلینجر) نیز انقباض ایجاد می‌کند؟
پاسخ مثبت است. در مقاله‌ی اخیر آسوده و ژانگ \cite{Asoodeh2024}، نشان داده شده است که مکانیزم‌های موضعی خاصیت «انقباض قوی» را برای طیف وسیعی از واگرایی‌ها به ارمغان می‌آورند.

\subsection{کران دقیق برای واگرایی کای-دو ($\chi^2$)}

یکی از مهم‌ترین نتایج این پژوهش، ارائه‌ی یک ضریب انقباض دقیق برای واگرایی $\ch^2$ است. اهمیت این واگرایی در آن است که کار با آن در محاسبات واریانس و کران‌های مینیماکس بسیار ساده‌تر از \lr{KL} است.

\begin{قضیه}[انقباض $\ch^2$]
\label{thm:chi2-contraction}
فرض کنید \mech\ یک مکانیزم \LDP باشد. برای هر دو توزیع ورودی $P$ و $Q$، واگرایی کای-دو بین توزیع‌های خروجی با رابطه زیر محدود می‌شود:
\begin{equation}
\label{eq:chi2-contraction}
\ch^2(\mech P || \mech Q) \le \et_\al \cdot \ch^2(P || Q)
\end{equation}
که در آن $\et_\al$ ضریب انقباض بهینه است و برابر است با:
\begin{equation}
\et_\al = \left( \frac{e^\al - 1}{e^\al + 1} \right)^2
\end{equation}
\end{قضیه}

\textbf{تحلیل مجانبی:}
برای مقادیر کوچک \al\ (رژیم محرمانگی بالا)، داریم:
\begin{equation}
\et_\al \approx \left( \frac{1 + \al - 1}{1 + \al + 1} \right)^2 \approx \left( \frac{\al}{2} \right)^2 = \frac{\al^2}{4}
\end{equation}
این نتیجه بسیار قابل توجه است. یادآوری می‌کنیم که کران‌های کلاسیک دوچی (فصل ۳) ضریبی از مرتبه $O(\al^2)$ داشتند، اما ضریب $1/4$ در اینجا نشان‌دهنده‌ی یک انقباض بسیار شدیدتر است. این ضریب دقیقاً با ضریب انقباض «پاسخ تصادفی دودویی» برای واریانس مطابقت دارد و نشان می‌دهد که این کران برای کل کلاس مکانیزم‌های \LDP «تایت» (Tight) است.

\subsection{تعمیم به سایر واگرایی‌ها}

نویسندگان در \cite{Asoodeh2024} نشان داده‌اند که این ضریب انقباض $\et_\al$ تنها مختص $\ch^2$ نیست، بلکه برای خانواده‌ای از واگرایی‌ها که خاصیت «تحدب مشترک» دارند (شامل فاصله هلینجر مجذور $H^2$ و واگرایی \lr{KL}) نیز صادق است.

\begin{نتیجه}
برای هر مکانیزم \LDP، کران‌های زیر برقرار هستند:
\begin{align}
D_{KL}(\mech P || \mech Q) &\le \et_\al \cdot D_{KL}(P || Q) \\
H^2(\mech P, \mech Q) &\le \et_\al \cdot H^2(P, Q)
\end{align}
این یکسان‌سازی ضرایب انقباض، تحلیل مکانیزم‌های پیچیده را بسیار ساده می‌کند؛ زیرا کافیست فقط ضریب $\et_\al$ را محاسبه کنیم.
\end{نتیجه}

\section{نامساوی ون‌تریز خصوصی (\lr{Private van Trees Inequality})}
\label{sec:new:van-trees}

اکثر تحلیل‌های موجود در ادبیات \LDP (مانند کارهای دوچی)، بر روی «ریسک مینیماکس» (بدترین حالت) تمرکز دارند. اما در بسیاری از کاربردهای مدرن، ما به تحلیل‌های بیزی (Bayesian) علاقه‌مندیم، جایی که پارامتر مجهول $\thh$ دارای یک توزیع پیشین $\pii(\thh)$ است.

نامساوی ون‌تریز (\lr{van Trees}) ابزاری کلاسیک برای کران‌دار کردن خطای بیزی بر اساس «اطلاعات فیشر» است. آسوده و ژانگ \cite{Asoodeh2024} نسخه‌ی خصوصی‌شده‌ی این نامساوی را ارائه کرده‌اند که ابزاری نوین در جعبه‌ابزار تحلیل محرمانگی محسوب می‌شود.

\begin{قضیه}[نامساوی ون‌تریز موضعی]
فرض کنید می‌خواهیم پارامتر $\thh$ را از روی مشاهدات $Z^n$ که خروجی یک مکانیزم \LDP هستند، تخمین بزنیم. اگر $\hat{\thh}$ هر تخمین‌گر دلخواهی باشد، آنگاه میانگین مربعات خطای بیزی\LTRfootnote{Bayesian Mean Square Error} دارای کران پایین زیر است:
\begin{equation}
\label{eq:pvt}
\Eset [(\hat{\thh} - \thh)^2] \ge \frac{1}{\Eset[\mathcal{I}(\thh)] + \mathcal{I}_{prior}(\pii)}
\end{equation}
نکته‌ی کلیدی اینجاست که در نسخه خصوصی، اطلاعات فیشر مشاهدات ($\mathcal{I}(\thh)$) با ضریب انقباض تضعیف می‌شود:
\begin{equation}
\mathcal{I}_{priv}(\thh) \le \et_\al \cdot \mathcal{I}_{orig}(\thh)
\end{equation}
که در آن $\mathcal{I}_{orig}$ اطلاعات فیشر داده‌های خام است.
\end{قضیه}

\textbf{تفسیر:}
این نامساوی به زبان ساده می‌گوید: «در دنیای \LDP، هر بیت اطلاعات فیشر که از داده‌ها می‌گیرید، به اندازه‌ی $\et_\al \approx \al^2/4$ تضعیف می‌شود.»
این نتیجه، اثبات حدود پایین برای مسائل تخمین پارامتر را بسیار ساده می‌کند. به جای درگیر شدن با لم‌های پیچیده‌ی اسود یا فانو، کافیست اطلاعات فیشر مسئله‌ی اصلی را محاسبه کنیم و در ضریب $\et_\al$ ضرب کنیم.

\section{کاربردهای نوین و بهبود نرخ‌ها}
\label{sec:new:refined-apps}

استفاده از کران‌های انقباض قوی (بخش \ref{sec:new:strong-contraction}) و نامساوی ون‌تریز خصوصی (بخش \ref{sec:new:van-trees}) منجر به بهبود نتایج در مسائل کلاسیک می‌شود.

به عنوان مثال، در مسئله‌ی \textbf{تخمین چگالی غیرپارامتری} برای کلاس توزیع‌های هموار (کلاس هولدر با پارامتر $\bet$)، استفاده از این ابزارهای جدید نشان می‌دهد که نرخ خطای بهینه دقیقاً برابر است با:
\begin{equation}
R_{opt} \asymp \left( \frac{1}{n \al^2} \right)^{\frac{2\bet}{2\bet + 1}}
\end{equation}
اگرچه مرتبه‌ی کلی نرخ همگرایی مشابه نتایج دوچی است، اما ضرایب ثابت بهبود یافته‌اند و مهم‌تر از آن، اثبات با استفاده از انقباض $\ch^2$ بسیار کوتاه‌تر و مستقیم‌تر از روش‌های مبتنی بر \lr{KL} است. این امر نشان‌دهنده‌ی برتری رویکرد مبتنی بر $E_\gam$ و انقباض قوی در تحلیل سیستم‌های محرمانگی تفاضلی است.