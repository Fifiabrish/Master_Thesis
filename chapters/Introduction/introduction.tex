‫
‫\chapter{مقدمه}
‫در این بخش به توضیح مسئله‌ی محرمانگی تفاضلی و اهمیت آن می‌پردازیم. سپس ادبیات موضوع را شرح داده و مسئله را بیان می‌کنیم.
‫
\section{اهمیت داده‌ها و ضرورت حفظ حریم خصوصی}
\label{sec:intro:importance}

در دهه‌های اخیر، جهان شاهد رشد انفجاری در تولید و جمع‌آوری داده‌ها بوده است.
پیشرفت‌های چشم‌گیر در فناوری‌های ذخیره‌سازی، محاسبات ابری و اینترنت اشیاء، منجر به انباشت حجم عظیمی از داده‌ها شده است که اغلب تحت عنوان کلان‌داده\LTRfootnote{Big Data} شناخته می‌شوند.
این داده‌ها سوخت اصلی موتورهای تصمیم‌گیری مدرن و سیستم‌های هوشمند هستند.
امروزه، الگوریتم‌های یادگیری ماشین\LTRfootnote{Machine Learning} و تحلیل داده\LTRfootnote{Data Analytics} با بهره‌گیری از این مخازن عظیم اطلاعاتی، قادرند الگوهای پیچیده‌ای را شناسایی کنند که در حوزه‌هایی نظیر تشخیص پزشکی، بهینه‌سازی ترافیک شهری، توصیه‌گرهای تجاری و سیاست‌گذاری‌های کلان اقتصادی کاربرد حیاتی دارند.

با این حال، این استفاده‌ی گسترده از داده‌ها، نگرانی‌های جدی و فزاینده‌ای را در خصوص حریم خصوصی\LTRfootnote{Privacy} افراد به وجود آورده است.
داده‌های خامی که برای آموزش مدل‌های هوشمند یا استخراج آماره‌ها استفاده می‌شوند، اغلب حاوی اطلاعات حساس\LTRfootnote{Sensitive Information} و شخصی هستند.
تاریخچه‌ی تراکنش‌های مالی، سوابق پزشکی، موقعیت‌های مکانی و حتی الگوهای جستجو در وب، همگی می‌توانند جزئیات دقیقی از زندگی خصوصی افراد را فاش کنند.
بنابراین، یک چالش اساسی شکل می‌گیرد: چگونه می‌توان از سودمندی\LTRfootnote{Utility} آماری داده‌ها بهره برد، بدون آنکه حریم خصوصی مشارکت‌کنندگان در داده‌ها نقض شود؟

در سال‌های ابتدایی عصر اطلاعات، تصور عمومی بر این بود که حذف شناسه‌های صریح\LTRfootnote{Explicit Identifiers} (مانند نام، کد ملی و شماره تلفن) برای محافظت از هویت افراد کافی است.
این فرایند که گمنام‌سازی \LTRfootnote{Anonymization} نامیده می‌شود، با این فرض انجام می‌شد که داده‌های باقی‌مانده قابلیت ردیابی به فرد خاصی را ندارند.
اما پژوهش‌های متعددی نشان داده‌اند که این روش‌های سنتی در برابر حملات بازشناسایی \LTRfootnote{Re-identification Attacks} به شدت آسیب‌پذیر هستند.
در این نوع حملات، مهاجم با استفاده از اطلاعات جانبی\LTRfootnote{Auxiliary Information} یا اتصال پایگاه‌داده‌های مختلف به یکدیگر، موفق به کشف هویت افراد در داده‌های به ظاهر گمنام می‌شود.

چندین رخداد مشهور در دو دهه‌ی گذشته، ناکارآمدی روش‌های سنتی گمنام‌سازی را اثبات کرده‌اند:

\begin{itemize}
  \item \textbf{داده‌های پزشکی ماساچوست:}
  در یکی از اولین و مشهورترین موارد، لاتانیا سوئینی نشان داد که می‌توان با ترکیب داده‌های پزشکی گمنام‌سازی شده (که نام بیماران از آن حذف شده بود) با فهرست عمومی رأی‌دهندگان، هویت افراد را بازشناسایی کرد.
  او با استفاده از ترکیب تاریخ تولد، جنسیت و کد پستی (که به آن‌ها شبه‌شناسه\LTRfootnote{Quasi-identifier} می‌گویند)، موفق شد پرونده پزشکی فرماندار وقت ایالت ماساچوست را شناسایی کند \cite{sweeney2002k}.

  \item \textbf{مجموعه داده‌ی نتفلیکس\LTRfootnote{Netflix Prize Data}:}
  شرکت نتفلیکس مجموعه‌ای از امتیازهای کاربران به فیلم‌ها را منتشر کرد که در آن شناسه‌های کاربری با اعداد تصادفی جایگزین شده بودند.
  پژوهشگران نشان دادند که با استفاده از اطلاعات عمومی موجود در وب‌سایت \lr{IMDb} و تطبیق الگوهای امتیازدهی، می‌توان هویت بسیاری از کاربران را با دقت بالا کشف کرد \cite{narayanan2008robust}.

  \item \textbf{داده‌های جستجوی \lr{AOL}:}
  در سال ۲۰۰۶، شرکت \lr{AOL} تاریخچه‌ی جستجوی هزاران کاربر خود را منتشر کرد.
  اگرچه نام کاربران حذف شده بود، اما تحلیل محتوای جستجوها منجر به شناسایی هویت افراد شد (از جمله پرونده مشهور تلما آرنولد) که نشان داد حتی خودِ داده‌ها نیز می‌توانند به عنوان شناسه عمل کنند \cite{barbaro2006face}.
\end{itemize}

این شواهد تجربی و نظری نشان می‌دهند که تعاریف هیوریستیک و روش‌های موردی (مانند حذف ستون‌ها یا مخدوش‌سازی ساده) نمی‌توانند تضمین امنیتی پایداری ارائه دهند.
مهاجمان همواره می‌توانند دانش پس‌زمینه‌ی پیش‌بینی‌نشده‌ای داشته باشند که مکانیزم‌های سنتی را دور بزند.
در نتیجه، نیاز مبرمی به یک چارچوب ریاضی دقیق احساس شد که بتواند حریم خصوصی را به صورت کمی تعریف کرده و تضمین دهد که ریسک افشای اطلاعات، مستقل از توان محاسباتی یا دانش جانبی مهاجم، همواره محدود باقی می‌ماند.
این نیاز، زمینه را برای ظهور مفهوم محرمانگی تفاضلی فراهم کرد که در بخش‌های آتی به تفصیل به آن خواهیم پرداخت.




\subsection{محرمانگی تفاضلی (\lr{DP})}
\label{sec:intro:cdp}

در پاسخ به چالش‌های امنیتی و ناکارآمدی روش‌های سنتی گمنام‌سازی، دوُرک و همکاران در سال ۲۰۰۶ مفهوم محرمانگی تفاضلی\LTRfootnote{Differential Privacy} را معرفی کردند \cite{dwork2006differential}.
این چارچوب ریاضی دقیق، به جای تمرکز بر ویژگی‌های ظاهری داده‌ها (مانند حذف نام‌ها)، بر فرایند تولید خروجی تمرکز دارد و تضمین می‌کند که حضور یا عدم حضور یک فرد خاص در پایگاه‌داده، تأثیر ناچیزی بر خروجی نهایی الگوریتم داشته باشد.

پیش از آنکه به تعاریف صوری و ریاضی بپردازیم، ضروری است که درک عمیقی از چیستی محرمانگی و تمایز بنیادین آن با مفاهیم امنیتی کلاسیک پیدا کنیم. بسیاری از سوتفاهم‌ها در این حوزه ناشی از تمایز ندادن دو مفهوم امنیت داده (که قلمرو رمزنگاری\LTRfootnote{Cryptography} است) و محرمانگی داده (که هدف ماست) می‌باشد. رمزنگاری اساساً سازوکاری برای کنترل دسترسی\LTRfootnote{Access Control} است و تضمین می‌کند که تنها افراد مجاز می‌توانند داده‌ها را ببینند؛ اما در برابر نشت اطلاعات از خروجی‌های مجاز سکوت می‌کند. تصور کنید یک پایگاه‌داده‌ی حساس پزشکی کاملاً رمزنگاری شده باشد و پژوهشگری مجاز، نتیجه‌ی یک تحلیل آماری ساده (مانند میانگین حقوق یا نرخ یک بیماری) را منتشر کند. رمزنگاری هیچ محافظتی در برابر استنتاج‌های ثانویه ارائه نمی‌دهد و مهاجم می‌تواند با ترکیب این خروجی مجاز با دانش پیشین\LTRfootnote{Auxiliary Knowledge} خود، اطلاعات خصوصی افراد را بازسازی کند. بنابراین، رمزنگاری شرط لازم است، اما برای حفظ محرمانگی کافی نیست؛ چرا که خودِ نتیجه‌ی تحلیل، حامل اطلاعات است.

در پاسخ به این چالش، دوُرک مفهوم محرمانگی را با ایده‌ی امنیت معنایی\LTRfootnote{Semantic Security} پیوند می‌زند. در این دیدگاه، هدف محرمانگی جلوگیری از یادگیری حقایق کلی درباره‌ی جامعه نیست، بلکه هدف محافظت از حقایق خاص مربوط به یک فرد مشخص است. فلسفه‌ی مرکزی این است که نتیجه‌ی هر تحلیلی باید تقریباً یکسان باشد، چه یک فرد خاص در آن مطالعه مشارکت کند و چه نکند. این تعریف، محرمانگی را به مفهوم ریسک گره می‌زند؛ به این معنا که مشارکت در یک پایگاه‌داده نباید باعث شود ریسک افشای رازهای یک فرد به طور چشم‌گیری افزایش یابد.

برای مدل‌سازی این مفهوم، می‌توانیم از استعاره‌ی جهان‌های موازی استفاده کنیم. دو پایگاه‌داده‌ی همسایه\LTRfootnote{Neighboring Database} (مانند \Dset و $\Dset'$) را به عنوان دو جهان موازی در نظر بگیرید که در یکی، داده‌های کاربر $x$ وجود دارد و در دیگری، این داده‌ها حذف یا تغییر یافته‌اند. هدف نهایی این است که از دیدگاه یک ناظر بیرونی (مهاجم)، این دو جهان غیرقابل تفکیک\LTRfootnote{Indistinguishable} باشند. اگر مکانیزم محرمانگی بتواند کاری کند که مهاجم با مشاهده‌ی خروجی، نتواند تشخیص دهد که این خروجی از کدام جهان آمده است، آنگاه حریم خصوصی کاربر $x$ حفظ شده است. 

برای رسیدن به این هدف، ما از الگوریتم‌های تصادفی\LTRfootnote{Randomized Algorithms} بهره می‌بریم. این الگوریتم‌ها با تزریق نویز کنترل‌شده، توزیع خروجی‌ها را بین دو مجموعه‌داده‌ی همسایه چنان به هم نزدیک می‌کنند که تمایز قائل شدن میان آن‌ها از نظر آماری ناممکن می‌شود.


\subsubsection{چرا روش‌های قطعی شکست می‌خورند؟}
دستیابی به هدف فوق با روش‌های قطعی\LTRfootnote{Deterministic} ممکن نیست. برای درک بهتر، یک حمله‌ی تفاضلی\LTRfootnote{Differencing Attack} کلاسیک را در نظر بگیرید. فرض کنید $f$ تابعی قطعی است که میانگین درآمد $n$ فرد در پایگاه‌داده را برمی‌گرداند ($f(\Dset) = \frac{1}{n}\sum_{i=1}^n x_i$). مهاجم می‌تواند دو پرس‌وجو انجام دهد:
\begin{enumerate}
 \item میانگین درآمد $n$ نفر حاضر در پایگاه‌داده $\left(M_1 = f(\Dset)\right)$.
 \item میانگین درآمد همان افراد، به جز فرد هدف $k$ $\left(M_2 = f(\Dset \setminus \{x_k\})\right)$.
\end{enumerate}
از آن‌جا که خروجی بدون نویز است، مهاجم با یک محاسبه‌ی ساده‌ی جبری 
$\left(x_k = n \cdot M_1 - (n-1) \cdot M_2\right)$،
مقدار دقیق درآمد فرد $k$ را به دست می‌آورد. این مثال نشان می‌دهد که هر تغییر کوچکی در ورودی یک تابع قطعی، به تغییری مشخص و قابل‌ردیابی در خروجی منجر می‌شود که بلافاصله دو جهان موازی را از هم متمایز می‌کند.

در واقع مثال میانگین را می‌توان به هر تابع قطعی
$f: \Xset^n \to \Zset$ تعمیم داد. فرض کنید مهاجم به دنبال بازیابی داده‌ی فرد $k$-ام ($x_k$) است. اگر سایر داده‌های موجود در پایگاه‌داده، یعنی $\Dset_{-k} = \{x_1, \dots, x_{k-1}, x_{k+1}, \dots, x_n\}$ برای مهاجم شناخته شده باشند (فرضی که در تحلیل‌های بدبینانه‌ی محرمانگی تفاضلی استاندارد است)، تابع خروجی را می‌توان تنها بر حسب متغیر مجهول $x_k$ به صورت $g(x) = f(x, \Dset_{-k})$ بازنویسی کرد.

اگر تابع $g$ روی دامنه‌ی $\Xset$ یک‌به‌یک\LTRfootnote{Injective} (یا حتی در بازه‌ای مشخص وارون‌پذیر) باشد، محرمانگی به طور کامل از بین می‌رود؛ زیرا مهاجم با مشاهده‌ی خروجی $z$، می‌تواند ورودی را به صورت $x_k = g^{-1}(z)$ بازیابی کند. حتی اگر $g$ کاملاً وارون‌پذیر نباشد، مشاهده‌ی $z$ فضای جستجوی مقادیر ممکن برای $x_k$ را به شدت کاهش می‌دهد:
\[
    x_k \in \{x \in \Xset \mid g(x) = z\}
\]

از دیدگاه نظریه اطلاعات، مشکل مکانیزم‌های قطعی این است که توزیع احتمال خروجی آن‌ها به ازای یک ورودی مشخص، یک جرم احتمالی\LTRfootnote{Probability Mass} تک‌نقطه‌ای (تابع دلتای دیراک) است. اگر دو پایگاه‌داده‌ی همسایه‌ی $\Dset$ و $\Dset'$ چنان باشند که $f(\Dset) \neq f(\Dset')$، آن‌گاه تکیه‌گاه\LTRfootnote{Support} توزیع‌های خروجی کاملاً مجزا خواهد بود. در نتیجه، واگرایی کولبک-لایبلر\LTRfootnote{Kullback-Leibler Divergence} بین آن‌ها بی‌نهایت می‌شود:
\[
    D_{KL}(\mech(\Dset) || \mech(\Dset')) = \infty
\]
این رابطه اثبات می‌کند که هیچ سطح محدودی از محرمانگی ($\eps < \infty$) با توابع قطعیِ غیرثابت قابل دستیابی نیست. بنابراین، همان‌طور که در ادبیات موضوع تأکید شده است \cite{dwork2014roth}، برای شکستن این وابستگی قطعی و ایجاد ابهام آماری، تصادفی‌سازی\LTRfootnote{Randomization} در فرآیند مکانیزم الزامی است.





\subsubsection{کاربردهای محرمانگی تفاضلی}
این چارچوب ریاضی امروزه به استاندارد طلایی در تحلیل داده‌های حساس تبدیل شده و کاربردهای آن فراتر از آمارهای ساده رفته است. برخی از مهم‌ترین کاربردهای آن عبارتند از:

\begin{itemize}
 \item \textbf{تخمین میانگین و مجموع\LTRfootnote{Mean and Sum Estimation}:} 
  اساسی‌ترین کاربرد \lr{DP} در محاسبه‌ی آماره‌های توصیفی است. سازمان‌های آماری (مانند اداره سرشماری آمریکا) از این روش برای انتشار میانگین درآمد، سن یا جمعیت مناطق استفاده می‌کنند، بدون آنکه داده‌های فردی شهروندان به خطر بیفتد.
  
  \item \textbf{انتشار هیستوگرام\LTRfootnote{Histogram Release}:} 
  بسیاری از تحلیل‌ها نیازمند دانستن توزیع داده‌ها هستند. \lr{DP} اجازه می‌دهد تا تعداد افراد در هر بازه (مثلاً گروه‌های سنی یا درآمدی) با دقت بالا منتشر شود، در حالی که نویز اضافه شده مانع از شناسایی افراد در گروه‌های کم‌جمعیت می‌شود.
  
  \item \textbf{یادگیری ماشین خصوصی\LTRfootnote{Private Machine Learning}:} 
  در آموزش مدل‌های عمیق، خطر به‌خاطرسپاری\LTRfootnote{Memorization} داده‌های آموزشی وجود دارد. با استفاده از الگوریتم‌هایی نظیر \lr{DP-SGD}، می‌توان مدل‌هایی آموزش داد که الگوهای کلی را یاد می‌گیرند اما قادر به بازتولید داده‌های آموزشی حساس (مانند تصاویر چهره یا متون خصوصی) نیستند.
  
  \item \textbf{سیستم‌های توصیه‌گر و داده‌های مکانی:} 
  شرکت‌های فناوری از \lr{DP} برای جمع‌آوری آمارهای رفتاری (مانند پربازدیدترین وب‌سایت‌ها یا مکان‌های پرتردد) استفاده می‌کنند تا بدون ردیابی لحظه‌ای کاربران، کیفیت خدمات خود را بهبود بخشند (مانند مکانیزم \lr{RAPPOR} در گوگل کروم).
\end{itemize}

این توضیحات، زیربنای اصلی تعاریف ریاضی دقیقی است که در فصل بعد به آن‌ها خواهیم پرداخت.




\subsection{محرمانگی تفاضلی موضعی (\lr{LDP})}
\label{sec:intro:ldp}


اگرچه محرمانگی تفاضلی متمرکز (\lr{CDP}) استاندارد طلایی حفاظت از داده‌ها محسوب می‌شود، اما پاشنه‌ی آشیل آن در فرضیه‌ی وجود یک متصدی مورد اعتماد\LTRfootnote{Trusted Curator} نهفته است که به تمام داده‌های خام دسترسی دارد. این مدل در دنیای واقعی با چالش‌های امنیتی و حقوقی جدی روبروست؛ چرا که تجربه نشان داده است اعتماد کامل به سرورهای مرکزی، حتی در صورت مدیریت توسط نهادهای بزرگ فناوری، همواره در معرض تهدید است. یکی از این خطرات، نفوذهای خارجی و سرقت انبوه داده‌هاست؛ به طوری که حتی پیشرفته‌ترین دیوارهای آتش\LTRfootnote{Firewalls} نیز در برابر حملات پیچیده آسیب‌پذیرند. در چنین شرایطی، اگر داده‌ها به صورت خام ذخیره شده باشند، نشت اطلاعاتی مانند آنچه در واقعه‌ی \lr{Equifax} رخ داد، مکانیزم‌های محرمانگی تفاضلی مرکزی را عملاً بی‌فایده می‌کند؛ زیرا مهاجم با دور زدن مکانیزم، مستقیماً به مخزن داده‌های حساس دست می‌یابد\cite{house2018equifax}. 

علاوه بر تهدیدهای خارجی، خطر سوءاستفاده‌های داخلی توسط کارمندان یا مدیران سیستم با دسترسی‌های سطح بالا نیز وجود دارد که امنیت داده‌ها را نه به ریاضیات، بلکه به اخلاق انسانی گره می‌زند. از سوی دیگر، محدودیت‌های حقوقی و احضاریه‌های قضایی نیز متصدی را ملزم به افشای اطلاعات می‌کند. در تمام این سناریوها، مدل متمرکز با یک نقطه شکست مرکزی\LTRfootnote{Single Point of Failure} روبروست. 

\subsubsection{گذار به مدل موضعی، حذف نیاز به اعتماد}

بنابراین، تنها راه تضمین قطعی حریم خصوصی، اتخاذ رویکردی است که در آن متصدی اساساً به داده‌های اصلی دسترسی نداشته باشد. این ضرورت، نقطه‌ی عزیمت ما از مدل متمرکز به سمت چارچوب محرمانگی تفاضلی موضعی (\lr{LDP}) است که در آن فرآیند خصوصی‌سازی پیش از خروج داده از دستگاه کاربر انجام می‌شود.

در این معماری، مرز اعتماد از سرور مرکزی به دستگاه شخصی کاربر (موبایل یا لپ‌تاپ) منتقل می‌شود. پروتکل به گونه‌ای طراحی می‌شود که هیچ‌کس، نه نفوذگران، نه کارمندان کنجکاو و نه حتی دولت‌ها، هرگز داده‌ی واقعی کاربر را مشاهده نکنند. سرور تنها نسخه‌هایی مخدوش و نویزدار از داده‌ها را دریافت می‌کند که به تنهایی بی‌معنی هستند، اما در تجمیع با تعداد زیادی داده‌ی دیگر، الگوهای آماری دقیق را آشکار می‌سازند. این رویکرد، خطر نقض حریم خصوصی را بسیار کنترل می‌کند.


\section{کارهای پیشین و مرور ادبیات}
\label{sec:intro:lit-review}
\subsection{آغازگرها: از پیمایش‌های آماری تا تعریف مدرن محرمانگی}
\label{sec:lit:foundations}

اگرچه نگرانی پیرامون محرمانگی داده‌ها قدمتی به اندازه خودِ آمار دارد، اما فرمول‌بندی ریاضی دقیق آن دستاورد قرن بیست و یکم است. ادبیات کلاسیک این حوزه با تلاش برای {کنترل افشای آماری}\LTRfootnote{Statistical Disclosure Control} آغاز شد، اما ناکارآمدی روش‌های مبتنی بر گمنام‌سازی در برابر دانش پس‌زمینه مهاجم، نیاز به یک تعریف معنایی قوی‌تر را ایجاب کرد.

نقطه عطف این تحول، معرفی مفهوم \textbf{محرمانگی تفاضلی}\LTRfootnote{Differential Privacy} توسط دِوُرک و همکاران بود \cite{dwork2006differential}. این تعریف، برخلاف روش‌های پیشین که بر ویژگی‌های داده تمرکز داشتند، بر ویژگی‌های مکانیزم پردازش داده تمرکز دارد. در مدل استاندارد (متمرکز)، یک مکانیزم تصادفی \mech \ دارای شرایط \DP \ است اگر برای هر دو پایگاه‌داده همسایه \Dset \ و $\Dset'$ (که تنها در داده یک فرد متفاوت‌اند) و برای هر زیرمجموعه از خروجی‌ها $\mathcal{S} \subseteq \mathrm{Range}(\mech)$، رابطه زیر برقرار باشد:

\begin{equation}
\label{eq:dp-def-lit}
\Pr[\mech(\Dset) \in \mathcal{S}] \le e^{\eps} \cdot \Pr[\mech(\Dset') \in \mathcal{S}] + \del
\end{equation}

که در آن \eps \ پارامتری کلیدی به نام \textbf{بودجه محرمانگی}\LTRfootnote{Privacy Budget} است و \del \ احتمال شکست ناچیز مکانیزم را نشان می‌دهد \cite{dwork2014roth}.
برای درک شهودی این مفهوم، می‌توان \eps \ را به عنوان یک «پیچ تنظیم» برای کنترل توازن میان امنیت و مطلوبیت داده‌ها در نظر گرفت. این پارامتر تعیین می‌کند که خروجی مکانیزم تا چه حد اجازه دارد بین دو جهان موازی (جهانی با حضور داده‌ی شما و جهانی بدون آن) تمایز قائل شود:
\begin{itemize}
    \item \textbf{مقادیر کوچک \eps \ (محرمانگی قوی):} زمانی که $\eps \to 0$، توزیع‌های خروجی برای دو پایگاه‌داده همسایه تقریباً بر هم منطبق می‌شوند. در این حالت، مکانیزم مجبور است نویز بسیار زیادی به پاسخ اضافه کند تا تفاوت‌ها را بپوشاند. در نتیجه، مهاجم تقریباً هیچ توانی برای تشخیص حضور فرد ندارد، اما در مقابل، دقت آماری خروجی کاهش می‌یابد.
    \item \textbf{مقادیر بزرگ \eps \ (محرمانگی ضعیف):} با افزایش \eps، مکانیزم آزادی عمل بیش‌تری دارد تا خروجی‌های متمایزتری تولید کند (نویز کمتر). این امر دقت تحلیل را افزایش می‌دهد، اما هم‌زمان ریسک بازشناسایی فرد و نشت اطلاعات خصوصی نیز به صورت نمایی بالا می‌رود.
\end{itemize}

همان‌طور که در بخش قبل توضیح داده شد، پیاده‌سازی این تعریف نیازمند یک پیش‌فرض قوی است: وجود یک {متصدی مورد اعتماد} که تمام داده‌های خام را جمع‌آوری کرده و نویز را به صورت مرکزی اعمال کند. اما دیدیم که این مدل دارای نقطه ضعف‌هایی است. حذف این فرض و انتقال اعتماد از سرور به کاربر، منجر به شکل‌گیری مفهوم \textbf{محرمانگی تفاضلی موضعی\LTRfootnote{Local Differential Privacy}(\lr{LDP})} شد. اگرچه اصطلاح \lr{LDP} و صورت‌بندی مدرن آن در سال‌های اخیر توسط پژوهشگرانی نظیر کاسی‌یسواناتان و دیگران تدوین شد \cite{whatcanwelearnprivatly}، اما ریشه‌های عملی آن به دهه‌ها قبل باز می‌گردد.


\begin{تعریف}[محرمانگی تفاضلی موضعی \lr{(\al,\del)-LDP}]
\label{def:ldp-approx-formal}
یک مکانیزم تصادفی $\mech: \Xset \to \Zset$ (تصادفی‌ساز موضعی) شرط «محرمانگی تفاضلی موضعی تقریبی» یا \ \LDP[(\al, \del)] را برآورده می‌کند، اگر برای تمام جفت ورودی‌های ممکن $x, x' \in \Xset$ و هر زیرمجموعه‌ی خروجی \ $\Sset \subseteq \Zset$، رابطه زیر برقرار باشد:
\begin{equation}
\label{eq:ldp-delta-def}
\Pr[\mech(x) \in \Sset] \le e^{\alpha} \cdot \Pr[\mech(x') \in \Sset] + \delta
\end{equation}
در این تعریف:
\begin{itemize}
    \item \al، بودجه محرمانگی است که میزان شباهت توزیع‌های خروجی را کنترل می‌کند.
    \item \del، احتمال ناچیز شکست مکانیزم در برقراری شرط محرمانگی است.
\end{itemize}
اگر $\delta = 0$ باشد، تعریف به حالت استاندارد یا «محرمانگی تفاضلی موضعی خالص» (\LDP) باز می‌گردد.
\end{تعریف}

در واقع، ساده‌ترین و نخستین نمونه از یک مکانیزم \lr{LDP}، روش «پاسخ تصادفی»\LTRfootnote{Randomized Response (RR)} است که توسط وارنر در سال ۱۹۶۵ برای حذف سوگیری در نظرسنجی‌های حساس معرفی شد \cite{warner1965randomized}. وارنر این روش را نه برای حفاظت در برابر حملات سایبری، بلکه برای تشویق پاسخ‌دهندگان به صداقت در سوالات حساس (مانند مصرف مواد مخدر یا عقاید سیاسی خاص) طراحی کرد.
    
    سازوکار کلاسیک این روش برای یک پرسش با پاسخ «بله/خیر» به صورت زیر است:
    فرض کنید از کاربر $i$ خواسته می‌شود که ویژگی حساس $X_i \in \{0, 1\}$ را گزارش کند. کاربر به جای پاسخ مستقیم، طبق دستورالعمل زیر عمل می‌کند:
    \begin{enumerate}
        \item یک سکه‌ را پرتاب می‌کند.  (می‌تواند سکه غیرمنصفانه\LTRfootnote{Unfair} باشد)
        \item اگر سکه «شیر» آمد، پاسخ واقعی ($X_i$) را گزارش می‌کند.
        \item اگر سکه «خط» آمد، یک پاسخ تصادفی (با پرتاب سکه‌ی دوم) تولید و گزارش می‌کند.
    \end{enumerate}
    در این سناریو، حتی اگر سرور پاسخ «بله» را دریافت کند، با قطعیت نمی‌داند که آیا کاربر واقعاً دارای ویژگی $X$ بوده است (شیر آمده) یا صرفاً به دلیل تصادف (خط آمدن سکه‌ی اول و شیر آمدن سکه‌ی دوم) این پاسخ را ارسال کرده است. با این حال، از آن‌جایی که احتمالات سکه‌ها مشخص است، سرور می‌تواند با جمع‌آوری تعداد زیادی از پاسخ‌ها ($n$ بسیار بزرگ)، اثر نویز را به صورت آماری حذف کرده و توزیع واقعی جامعه را با خطا تخمین بزند.
     به زبان ریاضی مدرن، اگر احتمال گزارش پاسخ واقعی $p$ باشد، نسبت احتمال خروجی‌ها برای دو ورودی متفاوت $x$ و $x'$ به صورت زیر محدود می‌شود:

\begin{equation}
\frac{\Pr[\mech(x)=z]}{\Pr[\mech(x')=z]} \le \frac{p}{1-p}
\end{equation}

این رابطه دقیقاً منطبق بر تعریف \LDP \ است و نشان می‌دهد که بودجه محرمانگی \al \ چگونه مستقیماً از پارامترهای مکانیزم ($p$) مشتق می‌شود:
\begin{equation}
e^\al = \frac{p}{1-p} \quad \Rightarrow \quad \al = \ln\left(\frac{p}{1-p}\right)
\end{equation}
این فرمول، تفسیر شهودی \lr{LDP} را کامل می‌کند:
\begin{itemize}
    \item اگر $p \approx 0.5$ (سکه کاملاً تصادفی)، آنگاه $\al \approx 0$ می‌شود. یعنی خروجی هیچ اطلاعاتی از ورودی ندارد (محرمانگی کامل، اما بدون فایده آماری).
    \item اگر $p \to 1$ (پاسخ تقریباً همیشه راست)، آنگاه $\al \to \infty$ می‌شود. یعنی داده‌ها دقیق هستند اما هیچ محرمانگی وجود ندارد.
\end{itemize}
بنابراین، کار وارنر را می‌توان سنگ‌بنای تاریخی این حوزه دانست که نشان داد چگونه می‌توان بدون اعتماد به گیرنده پیام، و با تنظیم دقیق پارامتر $p$ (و در نتیجه \al)، اطلاعات آماری مفیدی را مخابره کرد.

اما پاسخ تصادفی تنها راهکاری برای ایجاد محرمانگی در داده‌های دودویی است، و در کاربردهای مدرن با چالش دامنه‌ی بسیار بزرگ\LTRfootnote{High-Dimensional Domain} روبروست. شرکت‌های بزرگ فناوری نیاز دارند داده‌هایی نظیر «آدرس‌های اینترنتی بازدید شده» یا «کلمات جدید تایپ‌شده» را جمع‌آوری کنند که دامنه‌ی آن‌ها(\Xset) می‌تواند شامل میلیون‌ها حالت باشد. اعمال مستقیم \lr{RR} در این حالات منجر به نویز بسیار زیاد و کاهش شدید سودمندی می‌شود. در ادامه، راهکارهای اتخاذ شده توسط بزرگ‌ترین شرکت‌های فناوری را مرور می‌کنیم:

\begin{itemize}
    \item \textbf{گوگل و پروتکل \lr{RAPPOR}:}
    در سال ۲۰۱۴، گوگل برای جمع‌آوری آمار تنظیمات مرورگر کروم و شناسایی بدافزارها، پروتکل \lr{RAPPOR}\LTRfootnote{Randomized Aggregatable Privacy-Preserving Ordinal Response} را معرفی کرد \cite{Wang2020}. چالش اصلی گوگل، جمع‌آوری رشته‌های متنی\LTRfootnote{String} بود. راه‌حل آن‌ها ترکیب پاسخ تصادفی با فیلترهای بلوم\LTRfootnote{Bloom Filters} بود.
    در این روش، داده‌ی ورودی ابتدا به یک بردار بیتی (با استفاده از توابع درهم‌ساز) نگاشت می‌شود و سپس پاسخ تصادفی روی تک‌تک بیت‌های این فیلتر اعمال می‌گردد. این معماری به گوگل اجازه داد تا بدون دانستن ورودی دقیق هر کاربر، الگوهای پرتکرار و ناهنجاری‌ها را در مقیاس میلیونی شناسایی کند.

    \item \textbf{اپل و جمع‌آوری داده‌های دایره‌لغات:}
    شرکت اپل از \lr{LDP} برای بهبود کیبورد \lr{QuickType}، شناسایی ایموجی‌های پرطرفدار و داده‌های سلامت در سیستم‌عامل‌های \lr{iOS} و \lr{macOS} استفاده می‌کند. مسئله‌ی اپل، مخابره‌ی کارآمد داده‌ها با حفظ حریم خصوصی بود.
    راه‌حل اپل استفاده از تکنیک‌های مبتنی بر طرح‌ریزی\LTRfootnote{Sketching} و تبدیل‌های ریاضی مانند تبدیل هادامارد\LTRfootnote{Hadamard Transform} است. این تبدیل‌ها به مکانیزم اجازه می‌دهند که اطلاعات را در ابعاد پایین‌تر فشرده کند تا هم بار ارتباطی کاهش یابد و هم واریانس تخمین‌گر در دامنه‌های بزرگ کنترل شود \cite{Wang2020}.

    \item \textbf{مایکروسافت و داده‌های تله‌متری:}
    مایکروسافت برای جمع‌آوری داده‌های تله‌متری ویندوز (مانند مدت زمان استفاده از برنامه‌ها) با چالش تخمین هیستوگرام‌های پیوسته روبرو بود. آن‌ها از مکانیزم‌هایی نظیر نمونه‌برداری هیستوگرام و روش‌های تکرار‌کننده برای بازسازی توزیع داده‌ها استفاده کردند. تمرکز اصلی در این‌جا، ایجاد تعادل بین دقت آماری در جمع‌آوری داده‌های سیستمی و عدم امکان بازشناسایی رفتار یک کاربر خاص در طول زمان است.
\end{itemize}


این نمونه‌ها نشان می‌دهند که محرمانگی تفاضلی موضعی (\lr{LDP}) تنها یک مفهوم نظری نیست، بلکه یک ابزار حیاتی مهندسی است که با استفاده از تکنیک‌های پیشرفته‌ی آماری برای حل مسائل دنیای واقعی مقیاس‌دهی شده است.




\subsection{چالش سودمندی و موازنه دقت-محرمانگی}
\label{sec:lit:utility-challenge}

اگرچه حذف متصدی مرکزی در مدل \lr{LDP}، تضمین‌های امنیتی بسیار قوی‌تری را فراهم می‌کند، اما این امنیت رایگان به دست نمی‌آید. چالش بنیادین در این رویکرد، کاهش چشم‌گیر \textbf{سودمندی}\LTRfootnote{Utility} داده‌ها یا همان دقت تحلیل‌های آماری است. این پدیده تحت عنوان \textbf{موازنه محرمانگی-دقت}\LTRfootnote{Privacy-Accuracy Trade-off} شناخته می‌شود.

برای درک شهودی این چالش، مقایسه نحوه اعمال نویز در دو مدل ضروری است:
\begin{itemize}
    \item \textbf{در مدل متمرکز (\lr{CDP}):} نویز تنها «یک‌بار» و پس از تجمیع داده‌ها به نتیجه نهایی اضافه می‌شود. از آن‌جا که مجموع (یا میانگین) داده‌ها حساسیت کمی دارد، مقدار نویز معمولاً مستقل از تعداد کاربران ($n$) و بسیار کوچک است.
    \item \textbf{در مدل موضعی (\lr{LDP}):} نویز باید به «تک‌تک» داده‌های ورودی اضافه شود (پیش از آنکه از دستگاه کاربر خارج شوند). وقتی تحلیل‌گر قصد دارد میانگین این داده‌ها را محاسبه کند، واریانس نویزهای $n$ کاربر با هم جمع می‌شود.
\end{itemize}

این انباشت نویز باعث می‌شود که \textbf{نسبت سیگنال به نویز}\LTRfootnote{Signal-to-Noise Ratio (SNR)} در مدل موضعی بسیار پایین‌تر از مدل متمرکز باشد. به بیان دیگر، برای دستیابی به همان سطح از دقت که در مدل متمرکز وجود دارد، در مدل \lr{LDP} نیازمند تعداد بسیار بیش‌تری نمونه داده هستیم.

این مسئله در کاربردهای عملی بسیار حائز اهمیت است. برای مثال، اگر هدف تخمین فراوانی یک بیماری نادر باشد، نویز اضافه شده توسط مکانیزم‌های \lr{LDP} ممکن است سیگنال اصلی را کاملاً بپوشاند. همین چالش بود که پژوهشگران را بر آن داشت تا به جای استفاده از روش‌های ساده (مثل وارنر)، به دنبال پاسخ این پرسش باشند که: «آیا می‌توان مکانیزم‌هایی طراحی کرد که با کم‌ترین میزان نویز، بیش‌ترین محرمانگی را فراهم کنند؟» و «حد نهایی این دقت کجاست؟»
این پرسش‌ها زمینه را برای ورود تئوری‌های پیشرفته‌تر نظیر «تخمین مینیماکس» فراهم کرد.







\subsection{عصر مدرن \lr{LDP}: چارچوب مینیماکس و حدود بنیادین}
\label{sec:lit:modern-minimax}

پاسخ به پرسش بالا، مسیر پژوهش‌های این حوزه را به سمت \textbf{نظریه مینیماکس آماری}\LTRfootnote{Statistical Minimax Theory} تغییر داد. نقطه عطف این تحول، سلسله مقالات جریان‌ساز دوچی، جردن و وین‌رایت\LTRfootnote{Duchi, Jordan, and Wainwright} بود \cite{Duchi2013, duchi2018}. آن‌ها با صورتی‌بندی مسئله در قالب نظریه اطلاعات، نشان دادند که هزینه محرمانگی در مدل موضعی بسیار سنگین و غیرقابل اجتناب است.

در تحلیل مینیماکس، هدف یافتن «ریسک مینیماکس» ($\mathfrak{M}_n$) است؛ یعنی کم‌ترین خطایی که «بهترین تخمین‌گر ممکن» در «بدترین توزیع داده‌ی ممکن» مرتکب می‌شود. دوچی و همکاران با استفاده از ابزارهایی نظیر \textbf{نامساوی فانو}\LTRfootnote{Fano's Inequality} و \textbf{لم اسود}\LTRfootnote{Assouad's Lemma} (که در فصل سوم به تفصیل بررسی خواهند شد)، ثابت کردند که برای مسائل پایه‌ای نظیر تخمین میانگین یا چگالی احتمال، نرخ همگرایی خطا در مدل \lr{LDP} رفتاری متفاوت با مدل متمرکز دارد.

به طور مشخص، برای $n$ کاربر و بودجه محرمانگی $\al$، کران پایین خطا ($\cal{E}$) به صورت مجانبی از رابطه‌ی زیر پیروی می‌کند:

% \begin{equation}
\[
\label{eq:minimax-comparison}
\mathcal{E}_{LDP} \asymp \frac{1}{\sqrt{n \al^2}} \quad 
\text{در حالی که}
\quad \mathcal{E}_{CDP} \asymp \frac{1}{n \eps}
\]
% \end{equation}

این نتیجه که به «قانون مقیاس‌دهی کانونی»\LTRfootnote{Canonical Scaling Law} معروف است، حاوی دو پیام مهم است:
\begin{enumerate}
    \item \textbf{کندی همگرایی:} در حالی که خطای مدل متمرکز با سرعت $1/n$ کاهش می‌یابد، خطای مدل موضعی با سرعت بسیار کندتر $1/\sqrt{n}$ کم می‌شود.
    \item \textbf{اندازه نمونه مؤثر:} ضریب $\al^2$ نشان می‌دهد که هر نمونه داده‌ی خصوصی‌سازی شده، عملاً حاوی اطلاعاتی معادل با $\al^2$ نمونه داده‌ی خام است (برای $\al < 1$). این یعنی برای جبران نویز \LDP، حجم داده‌ها باید با ضریب $1/\al^2$ افزایش یابد.
\end{enumerate}

پس از استقرار این چارچوب نظری، تمرکز جامعه علمی بر طراحی «مکانیزم‌های بهینه ترتیب‌مقدماتی»\LTRfootnote{Order-optimal Mechanisms} قرار گرفت که بتوانند به این کران‌های نظری دست یابند.
از جمله مهم‌ترین این تلاش‌ها می‌توان به معرفی «مکانیزم‌های پله‌ای»\LTRfootnote{Staircase Mechanisms} توسط کایروز و همکاران \cite{Kairouz2016} و توسعه پروتکل‌های پیشرفته‌ای نظیر \lr{UE} (کدگذاری یگانی)\LTRfootnote{Unary Encoding} و \lr{OLH} (هشینگ محلی بهینه)\LTRfootnote{Optimal Local Hashing} توسط وانگ و همکاران \cite{Wang2020} اشاره کرد.
این روش‌ها تلاش می‌کنند با بهینه‌سازی ساختار نویز و استفاده از تکنیک‌های فشرده‌سازی اطلاعات، فاصله بین عملکرد عملی و حدود نظری مینیماکس را به حداقل برسانند.


\subsection{دسته‌بندی پروتکل‌های موضعی: تعاملی و غیرتعاملی}
\label{sec:lit:protocols-types}

به عنوان آخرین مبحث در مرور ادبیات موضوع، لازم است به دسته‌بندی پروتکل‌های \LDP \ بر اساس «معماری ارتباطی» اشاره کنیم. پژوهش‌های انجام شده در این حوزه، مکانیزم‌ها را به دو دسته‌ی کلی تقسیم می‌کنند:

\begin{enumerate}
    \item \textbf{پروتکل‌های غیرتعاملی}\LTRfootnote{Non-interactive / Simultaneous}:
    در این حالت، تمام کاربران به صورت هم‌زمان و مستقل عمل می‌کنند. هر کاربر $i$ مکانیزم \mech \ را تنها بر اساس داده‌ی خودش $X_i$ اجرا کرده و پیام $Z_i$ را به سرور می‌فرستد. هیچ ارتباطی بین کاربران وجود ندارد و سرور نیز هیچ بازخوردی به کاربران نمی‌دهد. به دلیل سادگی پیاده‌سازی و مقیاس‌پذیری بالا، اکثر پروتکل‌های صنعتی (مانند \lr{RAPPOR} گوگل یا سیستم‌های اپل) در این دسته قرار می‌گیرند.

    \item \textbf{پروتکل‌های تعاملی}\LTRfootnote{Interactive / Sequential}:
    در این روش، کاربران به صورت متوالی با سرور ارتباط برقرار می‌کنند. کاربر $i$ می‌تواند قبل از ارسال داده‌ی خود، خلاصه‌ای از داده‌های کاربران قبلی ($Z_1, \dots, Z_{i-1}$) را از سرور دریافت کند و نویز خود را هوشمندانه‌تر تنظیم نماید. اگرچه به نظر می‌رسد این آزادی عمل باید دقت را افزایش دهد، اما دوچی و همکاران در نتایج حیرت‌انگیزی نشان دادند که برای دسته‌ی بزرگی از توابع محدب (مانند تخمین میانگین)، پروتکل‌های تعاملی هیچ مزیتی نسبت به روش‌های غیرتعاملی ندارند و نرخ مینیماکس را بهبود نمی‌بخشند \cite{Duchi2013, Kairouz2016}.
\end{enumerate}


در این پایان‌نامه، چارچوب نظری ارائه‌شده به گونه‌ای است که نتایج (به‌ویژه کران‌های پایین مینیماکس) برای هر دو کلاس پروتکل‌های تعاملی و غیرتعاملی صادق هستند. ما نشان خواهیم داد که محدودیت‌های ذاتیِ مدل \lr{LDP}، مستقل از معماری ارتباطی شبکه عمل می‌کنند و افزودن تعامل، لزوماً راه گریزی از این محدودیت‌های بنیادین فراهم نمی‌کند. ابزارهای ریاضیاتی قدرتمندی که امکان چنین تحلیل یک‌پارچه‌ای را فراهم می‌سازند، در بخش بعدی معرفی خواهند شد.

‫

\section{بیان مسئله و اهداف پژوهش}
\label{sec:intro:problem-statement}

همان‌طور که در مرور ادبیات بیان شد، چارچوب مینیماکس ارائه شده توسط دوچی و همکاران \cite{Duchi2013}، نشان داد که محرمانگی تفاضلی موضعی (\LDP) منجر به کاهش چشمگیر نرخ همگرایی در تخمین‌های آماری می‌شود. با این حال، تحلیل‌های موجود در ادبیات اغلب به صورت موردی و با استفاده از ابزارهای پراکنده صورت گرفته است. برای مثال، کران‌های پایین معمولاً با استفاده از واگرایی کولبک-لایبلر (\lr{KL}) برای برخی مسائل و فاصله‌ی تغییرات کل (\lr{TV}) برای برخی دیگر اثبات می‌شوند.

این رویکرد چندگانه، دو چالش اصلی ایجاد می‌کند:
\begin{enumerate}
    \item \textbf{فقدان دیدگاه هندسی یکپارچه:} مشخص نیست که دقیقاً کدام ویژگی هندسیِ مکانیزم‌های \LDP \ مسئول کاهش اطلاعات است.
    \item \textbf{دشواری در تعمیم:} اثبات کران‌ها برای توابع زیان جدید یا توزیع‌های پیچیده، نیازمند ابداع تکنیک‌های اثباتی جدید است.
\end{enumerate}

مسئله‌ی اصلی این پژوهش، پر کردن این خلأ نظری از طریق توسعه‌ی یک چارچوب یکپارچه مبتنی بر نظریه اطلاعات است. ما به دنبال پاسخی جامع برای این پرسش هستیم: «چگونه می‌توان اتلاف اطلاعات در مکانیزم‌های \LDP \ را با استفاده از یک معیار عمومی سنجید و آن را مستقیماً به خطای تخمین آماری مرتبط کرد؟»

\subsection{رویکرد پژوهش: $f$-واگرایی‌ها به عنوان ابزار تحلیل}
\label{sec:intro:f-div-approach}

برای حل مسئله‌ی فوق، این پایان‌نامه پیشنهاد می‌کند که به جای تمرکز بر معیارهای خاص، از خانواده‌ی عمومی \textbf{$f$-واگرایی‌ها}\LTRfootnote{$f$-divergences} استفاده شود.
$f$-واگرایی‌ها (که توسط سیسار\LTRfootnote{Csiszár} معرفی شدند)، کلاسی از معیارهای فاصله بین دو توزیع احتمال $P$ و $Q$ هستند که به صورت زیر تعریف می‌شوند:
\begin{equation}
D_f(P \| Q) = \Eset_Q \left[ f\left( \frac{dP}{dQ} \right) \right]
\end{equation}
که در آن $f$ یک تابع محدب با ویژگی $f(1)=0$ است. اهمیت این کلاس در آن است که معیارهای مشهوری نظیر واگرایی \lr{KL}، فاصله‌ی هلینجر ($H^2$)، فاصله‌ی کای-دو ($\chi^2$) و فاصله‌ی تغییرات کل ($TV$) همگی حالت‌های خاصی از $f$-واگرایی هستند.

استفاده از این ابزار قدرتمند به ما اجازه می‌دهد تا مفهوم «محرمانگی» را به صورت یک محدودیت هندسی تفسیر کنیم. در این دیدگاه، مکانیزم \LDP \ به مثابه‌ی یک کانال انقباضی\LTRfootnote{Contraction Channel} عمل می‌کند که فاصله‌ی بین توزیع‌های ورودی را کاهش می‌دهد. هدف ما محاسبه‌ی دقیق این «ضریب انقباض» برای $f$-واگرایی‌های مختلف و استفاده از آن برای استخراج کران‌های مینیماکس است.

\subsection{نوآوری‌ها و مشارکت‌های پایان‌نامه}
\label{sec:intro:contributions}

این پژوهش با بهره‌گیری از ابزارهای فوق، مشارکت‌های زیر را در ادبیات موضوع ارائه می‌دهد:

\begin{itemize}
    \item \textbf{ارائه‌ی چارچوب یکپارچه برای تحلیل انقباض:}
    ما نشان می‌دهیم که چگونه محدودیت \LDP[\al] باعث محدود شدن طیف وسیعی از $f$-واگرایی‌ها می‌شود. به طور خاص، «نابرابری‌های قوی پردازش داده»\LTRfootnote{Strong Data Processing Inequalities (SDPI)} را برای مدل موضعی توسعه داده و کران‌های دقیقی برای ضریب انقباض در واگرایی‌های $\chi^2$ و \lr{KL} استخراج می‌کنیم.

    \item \textbf{تعمیم متدهای کران پایین (لوکام و فانو):}
    با جایگزینی معیارهای سنتی با $f$-واگرایی‌های عمومی، نسخه‌های تعمیم‌یافته‌ای از «متد دو نقطه‌ای لوکام»\LTRfootnote{Le Cam's Method} و «نامساوی فانو»\LTRfootnote{Fano's Inequality} را ارائه می‌دهیم. این ابزارها به ما امکان می‌دهند تا کران‌های پایین مینیماکس را برای دسته‌ی گسترده‌تری از مسائل تخمین با سهولت بیشتری اثبات کنیم.

    \item \textbf{تحلیل بهینگی مکانیزم‌ها:}
    با استفاده از هندسه‌ی $f$-واگرایی‌ها، نشان می‌دهیم که چرا مکانیزم‌های استاندارد (مانند لاپلاس) در ابعاد بالا ناکارآمد هستند و چرا خانواده‌ی مکانیزم‌های «پاسخ تصادفی تعمیم‌یافته» رفتار بهینه‌تری از خود نشان می‌دهند.
\end{itemize}


\section{ساختار  پایان‌نامه}
\label{sec:intro:structure}

ساختار ادامه‌ی این پایان‌نامه به شرح زیر سازمان‌دهی شده است:

\begin{itemize}
    \item \textbf{فصل دوم: مفاهیم اولیه}
    در این فصل، نمادگذاری‌های ریاضی و تعاریف پایه‌ی مورد نیاز برای فصول بعدی ارائه می‌شود. مفاهیم بنیادین نظریه اطلاعات، تعریف دقیق $f$-واگرایی‌ها و خواص آن‌ها، و همچنین صورت‌بندی رسمی مدل‌های محرمانگی تفاضلی (متمرکز و موضعی) در این فصل مرور خواهند شد.

    \item \textbf{فصل سوم: محرمانگی تفاضلی موضعی (\lr{LDP})}
    این فصل به بررسی عمیق‌تر چارچوب \LDP \ اختصاص دارد. ما مکانیزم‌های استاندارد این حوزه (نظیر پاسخ تصادفی و مکانیزم لاپلاس) را معرفی کرده و چالش‌های اساسی آن‌ها در ابعاد بالا را تحلیل می‌کنیم. همچنین، خواص انقباضی این مکانیزم‌ها از دیدگاه هندسی مورد بحث قرار می‌گیرد.

    \item \textbf{فصل چهارم: نرخ‌های مینیماکس و حدود پایین}
    این فصل هسته‌ی اصلی تحلیل‌های آماری پایان‌نامه را تشکیل می‌دهد. در این بخش، چارچوب تخمین مینیماکس معرفی شده و با استفاده از ابزار $f$-واگرایی‌ها، متدهای کلاسیک (نظیر روش لوکام و نامساوی فانو) بازنویسی می‌شوند. هدف نهایی این فصل، اثبات کران‌های پایین برای خطای تخمین در مسائل مختلف تحت قید \LDP \ است.

    \item \textbf{فصل پنجم: هم‌ارزی $f$-واگرایی‌ها و تحلیل‌های تکمیلی}
    در این فصل، روابط و هم‌ارزی‌های موجود میان $f$-واگرایی‌های مختلف در بستر محرمانگی موضعی بررسی می‌شود. نشان خواهیم داد که چگونه نتایج به دست آمده با یک معیار خاص، قابل تعمیم به سایر معیارها هستند و چه نتایجی از این هم‌ارزی‌ها حاصل می‌شود.

    \item \textbf{فصل ششم: نتیجه‌گیری و پیشنهادها}
    در نهایت، در فصل آخر ضمن مرور نتایج کلیدی پژوهش، به جمع‌بندی مباحث پرداخته و پیشنهادهایی برای پژوهش‌های آتی در این حوزه ارائه خواهیم داد.
\end{itemize}
‫
‫